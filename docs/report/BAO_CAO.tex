%!TeX program = xelatex
\documentclass[12pt,a4paper]{report}

% Packages
\usepackage{fontspec}
\setmainfont{Times New Roman}
\usepackage[vietnamese]{babel}
\usepackage[top=2cm, bottom=3cm, left=3cm, right=2cm]{geometry}
\usepackage{graphicx}
\usepackage{float}
\usepackage{hyperref}
\usepackage{indentfirst}
\usepackage{setspace}
\onehalfspacing
\usepackage{titlesec}
\usepackage{enumitem}

% Format headings
\titleformat{\chapter}[display]
  {\normalfont\bfseries\centering\fontsize{16}{19}\selectfont}
  {\chaptertitlename\ \thechapter}
  {10pt}
  {\fontsize{16}{19}\selectfont\MakeUppercase}
  
\titleformat{\section}
  {\normalfont\bfseries\fontsize{14}{16}\selectfont}
  {\thesection}
  {1em}
  {}

\titleformat{\subsection}
  {\normalfont\bfseries\fontsize{13}{15}\selectfont}
  {\thesubsection}
  {1em}
  {}

\titleformat{\subsubsection}
  {\normalfont\bfseries\fontsize{12}{14}\selectfont}
  {\thesubsubsection}
  {1em}
  {}

% Adjust chapter spacing
\titlespacing*{\chapter}{0pt}{-20pt}{20pt}

  {}
  
% Hyperlink setup
\hypersetup{
    colorlinks=true,
    linkcolor=black,
    filecolor=black,      
    urlcolor=blue,
    citecolor=black,
}

\begin{document}


\tableofcontents
\newpage

\chapter{TỔNG QUAN}

\section{Bối cảnh và vấn đề nghiên cứu}

\subsection{Thực trạng an ninh mạng trong hệ sinh thái IoT}

Trong những năm gần đây, Internet of Things (IoT) đã trở thành một phần không thể thiếu trong cuộc sống hiện đại, từ các thiết bị gia dụng thông minh, hệ thống y tế, đến các ứng dụng công nghiệp quy mô lớn. Theo dự báo của Statista, số lượng thiết bị IoT toàn cầu dự kiến đạt 75 tỷ vào năm 2025, tạo ra một hệ sinh thái số hóa rộng lớn nhưng cũng đi kèm với những thách thức lớn về bảo mật.

Tuy nhiên, phần lớn thiết bị IoT được thiết kế với tài nguyên tính toán hạn chế, thiếu cơ chế bảo mật mạnh mẽ, và thường sử dụng mật khẩu mặc định dễ đoán. Những điểm yếu này khiến chúng trở thành mục tiêu lý tưởng cho các cuộc tấn công mạng, đặc biệt là các botnet IoT -- mạng lưới thiết bị bị nhiễm mã độc và bị kiểm soát từ xa để thực hiện các hành vi phá hoại.

\subsection{Các cuộc tấn công botnet IoT điển hình}

Một trong những sự kiện đáng chú ý nhất là cuộc tấn công DDoS (Distributed Denial of Service) từ botnet Mirai vào năm 2016, đã làm tê liệt hàng loạt website lớn như Twitter, Netflix, Reddit bằng cách khai thác hơn 600.000 thiết bị IoT bị nhiễm mã độc \cite{ref17, ref18}. Sự kiện này đã gióng lên hồi chuông cảnh báo về mức độ nghiêm trọng của mối đe dọa từ botnet IoT.

Các dạng tấn công phổ biến từ botnet IoT bao gồm:

\begin{itemize}
    \item \textbf{DDoS (Distributed Denial of Service)}: Tấn công từ nhiều nguồn khác nhau nhằm làm quá tải hệ thống mục tiêu.
    \item \textbf{DoS (Denial of Service)}: Tấn công từ một hoặc vài nguồn để ngăn chặn dịch vụ hợp pháp.
    \item \textbf{Reconnaissance (Trinh sát)}: Thu thập thông tin về mạng và thiết bị để chuẩn bị cho các cuộc tấn công tiếp theo.
    \item \textbf{Theft (Trộm cắp dữ liệu)}: Đánh cắp thông tin nhạy cảm từ thiết bị bị xâm nhập.
\end{itemize}

\subsection{Thách thức trong phát hiện và phòng chống}

Việc phát hiện và phân loại các cuộc tấn công botnet IoT gặp phải nhiều thách thức lớn:

\textbf{Thách thức về dữ liệu}: Tập dữ liệu trong thực tế thường có sự mất cân bằng nghiêm trọng (class imbalance), trong đó lưu lượng tấn công chiếm tỷ lệ áp đảo so với lưu lượng bình thường. Ví dụ, trong tập dữ liệu Bot-IoT \cite{ref1} được sử dụng trong nghiên cứu này, lưu lượng tấn công chiếm tới 99.95\% trong khi lưu lượng bình thường chỉ chiếm khoảng 0.05\%. Sự mất cân bằng này khiến các mô hình học máy truyền thống dễ bị thiên lệch về lớp đa số, dẫn đến khả năng phát hiện lưu lượng bình thường kém.

\textbf{Thách thức về độ phức tạp}: Các cuộc tấn công ngày càng tinh vi với nhiều kỹ thuật ẩn náu và mã hóa, khiến việc phân biệt giữa lưu lượng tấn công và lưu lượng bình thường trở nên khó khăn hơn. Đặc biệt, việc phân biệt giữa DDoS (nhiều nguồn tấn công) và DoS (ít nguồn tấn công) đòi hỏi phân tích sâu về đặc trưng nguồn tấn công (source diversity).

\textbf{Thách thức về quy mô}: Với hàng tỷ thiết bị IoT hoạt động đồng thời, khối lượng dữ liệu lưu lượng mạng cần phân tích là rất lớn. Điều này đặt ra yêu cầu cao về khả năng mở rộng (scalability) và hiệu năng xử lý của hệ thống phát hiện xâm nhập.

\textbf{Thách thức về tài nguyên}: Nhiều giải pháp hiện tại yêu cầu tài nguyên tính toán lớn, không phù hợp với các tổ chức nhỏ hoặc môi trường giáo dục có ngân sách hạn chế.

\subsection{Tổng quan các nghiên cứu liên quan}

Nhiều nghiên cứu đã áp dụng các kỹ thuật học máy và học sâu để giải quyết vấn đề an ninh trong IoT. Các bài khảo sát của Buczak et al. \cite{ref19} và Khraisat et al. \cite{ref20} đã tổng hợp các kỹ thuật phát hiện xâm nhập phổ biến.

Về dữ liệu, Sarhan et al. \cite{ref2} đã phân tích và so sánh các bộ dữ liệu NetFlow cho ML-based IDS. Koroniotis et al. \cite{ref1} công bố bộ dữ liệu Bot-IoT, khắc phục nhiều hạn chế của các bộ dữ liệu cũ.

Về phương pháp, Alosaimi và Almutairi \cite{ref3} áp dụng các kỹ thuật học máy cơ bản trên Bot-IoT. Zawaideh et al. \cite{ref4} đề xuất kết hợp CNN và XGBoost. Zhang et al. \cite{ref5} và Özdoğan et al. \cite{ref6} đã phát triển các mô hình hai giai đoạn (two-stage) để nâng cao hiệu quả phát hiện. Các nghiên cứu khác như Le et al. \cite{ref7}, Doghramachi và Ameen \cite{ref8}, Chalichalamala et al. \cite{ref9}, và Khan et al. \cite{ref10} tập trung vào việc tối ưu hóa XGBoost và xử lý mất cân bằng dữ liệu trong môi trường IoT tài nguyên hạn chế.

Tuy nhiên, các phương pháp này thường chưa giải quyết triệt để vấn đề phân loại đa lớp chi tiết trong điều kiện mất cân bằng dữ liệu cực đoan (extreme imbalance), điều mà nghiên cứu này tập trung giải quyết.

\section{Tính cấp thiết của đề tài}

\subsection{Khoảng trống nghiên cứu}

Mặc dù đã có nhiều nghiên cứu về phát hiện tấn công mạng sử dụng học máy, vẫn tồn tại những khoảng trống quan trọng cần được giải quyết:

\textbf{Xử lý mất cân bằng dữ liệu nghiêm trọng}: Hầu hết các nghiên cứu hiện tại chỉ xử lý mất cân bằng ở mức độ trung bình hoặc sử dụng tập dữ liệu đã được cân bằng trước. Tuy nhiên, trong thực tế, tỷ lệ mất cân bằng có thể lên tới 2000:1 hoặc cao hơn, đòi hỏi các kỹ thuật xử lý chuyên sâu hơn như SMOTE (Synthetic Minority Over-sampling Technique) \cite{ref13} và cân bằng trọng số trong mô hình \cite{ref14, ref15}.

\textbf{Phân loại đa cấp hiệu quả}: Nhiều nghiên cứu chỉ tập trung vào phân loại nhị phân (tấn công/bình thường) hoặc phân loại đa lớp trực tiếp. Trong khi đó, mô hình phân cấp hai giai đoạn (hierarchical two-stage) \cite{ref5, ref16} có tiềm năng cải thiện độ chính xác bằng cách chia nhỏ bài toán phức tạp thành các bước đơn giản hơn.

\textbf{Tối ưu hóa tài nguyên}: Các giải pháp thường yêu cầu phần cứng chuyên dụng hoặc máy chủ mạnh mẽ. Nghiên cứu này hướng tới giải pháp có thể triển khai trên các nền tảng điện toán đám mây miễn phí hoặc chi phí thấp như Google Colab \cite{ref23}, phù hợp với môi trường giáo dục và nghiên cứu.

\subsection{Hạn chế của các phương pháp hiện tại}

Các phương pháp phát hiện xâm nhập truyền thống như signature-based detection chỉ có thể phát hiện các cuộc tấn công đã biết, trong khi anomaly-based detection thường có tỷ lệ false positive cao. Các mô hình học máy đơn giản như Decision Tree, Random Forest tuy có độ chính xác tốt nhưng gặp khó khăn khi xử lý dữ liệu mất cân bằng nghiêm trọng.

Mô hình Deep Learning như CNN, LSTM mặc dù cho kết quả tốt nhưng yêu cầu tài nguyên tính toán lớn, thời gian huấn luyện dài, và khó giải thích được quyết định của mô hình (lack of interpretability) \cite{ref21, ref22}. Điều này khiến chúng không phù hợp cho các ứng dụng thời gian thực hoặc môi trường có tài nguyên hạn chế.

\section{Mục tiêu nghiên cứu}

\subsection{Mục tiêu tổng quát}

Xây dựng hệ thống phát hiện và phân loại botnet IoT dựa trên mô hình học máy phân cấp hai giai đoạn (Two-Stage Hierarchical Model), có khả năng xử lý hiệu quả dữ liệu mất cân bằng nghiêm trọng, đạt độ chính xác cao và có thể triển khai trên nền tảng điện toán đám mây với tài nguyên hợp lý.

\subsection{Mục tiêu cụ thể}

Để đạt được mục tiêu tổng quát, nghiên cứu đặt ra các mục tiêu cụ thể sau:

\begin{enumerate}
    \item \textbf{Xây dựng pipeline xử lý dữ liệu quy mô lớn}: Phát triển quy trình xử lý, gộp và phân tích tập dữ liệu Bot-IoT (74 file CSV, khoảng 16GB) một cách hiệu quả, tận dụng tối đa thông tin từ dữ liệu thực tế.
    \item \textbf{Thiết kế mô hình phân cấp hai giai đoạn}:
    \begin{itemize}
        \item Giai đoạn 1: Phân loại nhị phân (Binary Classification) để phân biệt lưu lượng tấn công và lưu lượng bình thường.
        \item Giai đoạn 2: Phân loại đa lớp (Multi-class Classification) để xác định loại tấn công cụ thể (DDoS, DoS, Reconnaissance).
    \end{itemize}
    \item \textbf{Xử lý mất cân bằng dữ liệu}: Áp dụng kỹ thuật SMOTE để tạo dữ liệu tổng hợp cho lớp thiểu số và sử dụng cân bằng trọng số (scale\_pos\_weight) trong XGBoost để cải thiện khả năng phát hiện lưu lượng bình thường.
    \item \textbf{Tối ưu hóa đặc trưng}: Thiết kế các đặc trưng phân biệt nguồn tấn công (source diversity features) bao gồm unique\_src\_count, src\_entropy, và top\_src\_ratio để phân biệt hiệu quả giữa DDoS và DoS.
    \item \textbf{Đánh giá hiệu năng toàn diện}: Thực hiện đánh giá mô hình trên nhiều chỉ số khác nhau (accuracy, precision, recall, F1-score) cho cả hai giai đoạn và đánh giá hiệu năng tổng thể của pipeline.
    \item \textbf{Tối ưu hóa tài nguyên và thời gian}: Đảm bảo hệ thống có thể chạy trên Google Colab Pro+ với cấu hình 52GB RAM và GPU T4/V100, hoàn thành quá trình huấn luyện trong khoảng 15-20 phút.
\end{enumerate}

\section{Phạm vi nghiên cứu và đối tượng nghiên cứu}

\subsection{Phạm vi nghiên cứu}

Nghiên cứu tập trung vào các khía cạnh sau:

\textbf{Phạm vi về dữ liệu}: Sử dụng tập dữ liệu Bot-IoT \cite{ref1} được công bố bởi UNSW Canberra Cyber, bao gồm 74 file CSV với tổng dung lượng khoảng 16GB, chứa khoảng 40-50 triệu bản ghi lưu lượng mạng. Tập dữ liệu này bao gồm cả lưu lượng bình thường và các loại tấn công botnet IoT thực tế.

\textbf{Phạm vi về phương pháp}: Tập trung vào kỹ thuật học máy gradient boosting, cụ thể là XGBoost, kết hợp với các kỹ thuật xử lý mất cân bằng dữ liệu như SMOTE và scale\_pos\_weight. Không nghiên cứu các phương pháp Deep Learning phức tạp do hạn chế về tài nguyên.

\textbf{Phạm vi về loại tấn công}: Nghiên cứu tập trung vào ba loại tấn công chính: DDoS, DoS và Reconnaissance. Loại tấn công Theft có số lượng mẫu quá ít sẽ được loại bỏ khỏi quá trình huấn luyện.

\textbf{Phạm vi về môi trường triển khai}: Hệ thống được thiết kế và tối ưu hóa cho nền tảng Google Colab Pro+ với cấu hình High-RAM (52GB) và GPU (T4/V100/A100), phù hợp với môi trường giáo dục và nghiên cứu.

\subsection{Đối tượng nghiên cứu}

\textbf{Đối tượng chính}: Các luồng lưu lượng mạng (network flows) trong môi trường IoT, được biểu diễn dưới dạng các đặc trưng thống kê như số lượng gói tin, số byte, thời lượng kết nối, các giá trị thống kê (mean, stddev, min, max), và các đặc trưng về nguồn tấn công.

\textbf{Đối tượng phụ}: Các kỹ thuật xử lý mất cân bằng dữ liệu, phương pháp tối ưu hóa siêu tham số, và chiến lược tận dụng GPU để tăng tốc quá trình huấn luyện.

\section{Hướng tiếp cận và phương pháp thực hiện}

\subsection{Hướng tiếp cận tổng quát}

Nghiên cứu áp dụng hướng tiếp cận phân cấp hai giai đoạn (hierarchical two-stage approach) với các nguyên tắc chính:

\textbf{Chia nhỏ bài toán phức tạp}: Thay vì xây dựng một mô hình phân loại đa lớp trực tiếp, nghiên cứu chia thành hai giai đoạn:
\begin{itemize}
    \item Giai đoạn 1 tập trung vào việc phân biệt có tấn công hay không (binary classification).
    \item Giai đoạn 2 chỉ xử lý các mẫu được xác định là tấn công để phân loại loại tấn công cụ thể (multi-class classification).
\end{itemize}

\textbf{Tối ưu hóa từng giai đoạn}: Mỗi giai đoạn được tối ưu hóa riêng biệt với các siêu tham số và kỹ thuật xử lý mất cân bằng phù hợp, giúp tăng hiệu năng tổng thể của hệ thống.

\textbf{Tận dụng GPU}: Sử dụng XGBoost với cấu hình tree\_method="gpu\_hist" và predictor="gpu\_predictor" để tận dụng tối đa GPU, giảm thời gian huấn luyện từ hàng giờ xuống còn 15-20 phút.

\subsection{Phương pháp thực hiện}

Nghiên cứu thực hiện theo các bước chính:

\textbf{Bước 1 - Tiền xử lý dữ liệu}:
\begin{itemize}
    \item Gộp 74 file CSV thành 8 batch file (mỗi batch khoảng 10 file), giảm số lần đọc/ghi file.
    \item Phân tích thống kê từng batch để xác định phân phối lớp và chọn batch phù hợp cho huấn luyện.
    \item Tạo tập test cân bằng (balanced test set) để đánh giá mô hình một cách công bằng.
\end{itemize}

\textbf{Bước 2 - Kỹ thuật đặc trưng (Feature Engineering)}:
\begin{itemize}
    \item Trích xuất các đặc trưng cơ bản từ lưu lượng mạng (protocol, flags, state, packets, bytes, duration).
    \item Thiết kế đặc trưng phân biệt nguồn (source diversity features) dựa trên thống kê địa chỉ nguồn trong cửa sổ thời gian:
    \begin{itemize}
        \item unique\_src\_count: Số lượng địa chỉ IP nguồn duy nhất
        \item src\_entropy: Entropy của phân phối địa chỉ nguồn
        \item top\_src\_ratio: Tỷ lệ của địa chỉ nguồn xuất hiện nhiều nhất
    \end{itemize}
\end{itemize}

\textbf{Bước 3 - Xử lý mất cân bằng}:
\begin{itemize}
    \item Áp dụng SMOTE cho lớp thiểu số trong cả hai giai đoạn.
    \item Sử dụng scale\_pos\_weight trong XGBoost để tăng trọng số cho lớp thiểu số.
\end{itemize}

\textbf{Bước 4 - Huấn luyện mô hình}:
\begin{itemize}
    \item Huấn luyện Stage 1 với XGBoost để phân loại nhị phân (Attack vs Normal).
    \item Huấn luyện Stage 2 với XGBoost để phân loại đa lớp (DDoS vs DoS vs Reconnaissance).
    \item Sử dụng early stopping để tránh overfitting.
\end{itemize}

\textbf{Bước 5 - Đánh giá và tối ưu}:
\begin{itemize}
    \item Đánh giá mô hình trên tập test cân bằng.
    \item Phân tích ma trận nhầm lẫn (confusion matrix) để xác định điểm mạnh và điểm yếu.
    \item Tạo các biểu đồ trực quan hóa để đánh giá hiệu năng.
\end{itemize}

\section{Ý nghĩa khoa học và thực tiễn}

\subsection{Ý nghĩa khoa học}

\textbf{Đóng góp về mô hình}: Nghiên cứu đề xuất kiến trúc phân cấp hai giai đoạn kết hợp với kỹ thuật xử lý mất cân bằng đa cấp, có thể được áp dụng cho các bài toán phân loại có mất cân bằng nghiêm trọng tương tự trong lĩnh vực an ninh mạng.

\textbf{Đóng góp về đặc trưng}: Các đặc trưng source diversity (unique\_src\_count, src\_entropy, top\_src\_ratio) được thiết kế dựa trên phân tích đặc tính tấn công DDoS và DoS, có thể được sử dụng để cải thiện khả năng phân biệt giữa hai loại tấn công này trong các nghiên cứu khác.

\textbf{Đóng góp về phương pháp}: Quy trình xử lý dữ liệu quy mô lớn (74 file, 16GB) một cách hiệu quả trên nền tảng có tài nguyên hạn chế, cung cấp hướng dẫn thực tiễn cho các nhà nghiên cứu gặp phải vấn đề tương tự.

\subsection{Ý nghĩa thực tiễn}

\textbf{Cho lĩnh vực giáo dục}: Hệ thống có thể được sử dụng làm công cụ giảng dạy và học tập về an ninh mạng IoT, giúp sinh viên hiểu rõ hơn về các kỹ thuật phát hiện xâm nhập hiện đại. Code được tổ chức rõ ràng, có chú thích chi tiết, dễ dàng tái sử dụng và mở rộng.

\textbf{Cho các tổ chức nhỏ và vừa}: Giải pháp có thể triển khai trên nền tảng điện toán đám mây với chi phí thấp, không yêu cầu đầu tư phần cứng chuyên dụng, phù hợp với các tổ chức có ngân sách hạn chế nhưng vẫn muốn có hệ thống phát hiện tấn công hiệu quả.

\textbf{Cho nghiên cứu tiếp theo}: Kết quả của nghiên cứu này có thể là nền tảng để phát triển các hệ thống phát hiện xâm nhập thời gian thực, tích hợp vào các thiết bị IoT gateway hoặc các giải pháp Network Security Monitoring (NSM).

\textbf{Cho cộng đồng mã nguồn mở}: Toàn bộ code và quy trình được công khai trên GitHub, cho phép cộng đồng nghiên cứu tái sử dụng, kiểm chứng và cải thiện, góp phần thúc đẩy sự phát triển của lĩnh vực an ninh IoT.

\section{Cấu trúc báo cáo}

Báo cáo được tổ chức thành năm chương chính:

\textbf{Chương 1 - Tổng quan}: Trình bày bối cảnh nghiên cứu, vấn đề cần giải quyết, tính cấp thiết, mục tiêu, phạm vi nghiên cứu, hướng tiếp cận tổng quát và ý nghĩa của đề tài.

\textbf{Chương 2 - Mô hình đề xuất}: Trình bày kiến trúc hệ thống Two-Stage Hierarchical Model, các thuật toán chính, quy trình hoạt động chi tiết và phân tích độ phức tạp.

\textbf{Chương 3 - Thực nghiệm và thảo luận}: Mô tả môi trường thực nghiệm, kết quả huấn luyện, đánh giá hiệu năng mô hình, so sánh với các phương pháp khác và thảo luận kết quả.

\textbf{Chương 4 - Kết luận và hướng phát triển}: Tóm tắt các đóng góp chính của nghiên cứu, đánh giá mức độ đạt được các mục tiêu đã đề ra, chỉ ra các hạn chế cần khắc phục, và đề xuất các hướng nghiên cứu tiếp theo.

\textbf{Tài liệu tham khảo}: Liệt kê các tài liệu tham khảo trong quá trình nghiên cứu.

\chapter{MÔ HÌNH ĐỀ XUẤT}

\section{Kiến trúc tổng thể hệ thống}

\subsection{Sơ đồ tổng quan}

Hệ thống phát hiện và phân loại botnet IoT được thiết kế theo kiến trúc phân cấp hai giai đoạn (Two-Stage Hierarchical Model), trong đó mỗi giai đoạn chuyên biệt hóa cho một tác vụ phân loại cụ thể. Kiến trúc này cho phép tối ưu hóa hiệu năng của từng giai đoạn một cách độc lập và giảm độ phức tạp của bài toán tổng thể.

\begin{figure}[H]
    \centering
    \includegraphics[width=0.95\textwidth]{figures/system_overview.png}
    \caption{Sơ đồ luồng dữ liệu của hệ thống Three-Stage Hierarchical Model}
    \label{fig:architecture}
\end{figure}

\subsection{Các thành phần chính}

Hệ thống bao gồm năm thành phần chính:

\textbf{Thành phần 1 - Tiền xử lý dữ liệu (Data Preprocessing)}:
\begin{itemize}
    \item Chức năng: Gộp 74 file CSV thành 8 batch files, phân tích thống kê và chọn batch phù hợp cho huấn luyện.
    \item Input: 74 file CSV gốc từ tập dữ liệu Bot-IoT
    \item Output: 8 batch files và file thống kê JSON
    \item Công nghệ: Pandas DataFrame operations, JSON serialization
\end{itemize}

\textbf{Thành phần 2 - Kỹ thuật đặc trưng (Feature Engineering)}:
\begin{itemize}
    \item Chức năng: Trích xuất 22 đặc trưng từ dữ liệu thô, bao gồm 3 đặc trưng source diversity đặc biệt để phân biệt DDoS và DoS.
    \item Input: Raw network flows với 35+ columns
    \item Output: 22 engineered features
    \item Công nghệ: Scipy entropy, NumPy aggregation, time-window analysis
\end{itemize}

\textbf{Thành phần 3 - Stage 1: Binary Classifier}:
\begin{itemize}
    \item Chức năng: Phân loại nhị phân (Attack vs Normal)
    \item Input: 22 features
    \item Output: is\_attack (0 = Normal, 1 = Attack)
    \item Mô hình: XGBoost Binary Classifier với SMOTE và scale\_pos\_weight
    \item Metrics: Accuracy, Precision, Recall, F1-Score, ROC-AUC
\end{itemize}

\textbf{Thành phần 4 - Stage 2: Multi-class Classifier}:
\begin{itemize}
    \item Chức năng: Phân loại đa lớp cho các mẫu được xác định là tấn công
    \item Input: 22 features (chỉ attack samples)
    \item Output: attack\_type (0 = DDoS, 1 = DoS, 2 = Reconnaissance)
    \item Mô hình: XGBoost Multi-class Classifier với SMOTE và balanced sample weights
    \item Metrics: Accuracy, Precision, Recall, F1-Score (weighted average)
\end{itemize}

\textbf{Thành phần 5 - Pipeline Integration}:
\begin{itemize}
    \item Chức năng: Kết hợp dự đoán từ hai giai đoạn để đưa ra quyết định cuối cùng
    \item Logic: Nếu Stage 1 dự đoán Normal $\to$ Output "Normal", nếu Attack $\to$ truyền qua Stage 2 để xác định loại tấn công cụ thể
    \item Output: Final classification (Normal / DDoS / DoS / Reconnaissance)
\end{itemize}

\section{Các thuật toán và mô-đun chính}

\subsection{Thuật toán XGBoost (Extreme Gradient Boosting)}

XGBoost là thuật toán học máy dựa trên gradient boosting \cite{ref11, ref12}, xây dựng một ensemble của nhiều cây quyết định yếu (weak learners) theo cách tuần tự, trong đó mỗi cây mới được huấn luyện để sửa lỗi của các cây trước đó.

\textbf{Hàm mục tiêu}:
\begin{equation}
    L(\phi) = \sum l(\hat{y}_i, y_i) + \sum \Omega(f_k)
\end{equation}

Trong đó:
\begin{itemize}
    \item $l(\hat{y}_i, y_i)$: Loss function (logloss cho binary, mlogloss cho multi-class)
    \item $\Omega(f_k)$: Regularization term để tránh overfitting
    \item $\phi$: Tập hợp các tham số của model
\end{itemize}

\textbf{Regularization term}:
\begin{equation}
    \Omega(f) = \gamma T + \frac{1}{2}\lambda \sum w_j^2
\end{equation}

Trong đó:
\begin{itemize}
    \item $T$: Số lượng lá (leaves) trong cây
    \item $w_j$: Trọng số của lá thứ j
    \item $\gamma$: Complexity control parameter
    \item $\lambda$: L2 regularization parameter
\end{itemize}

\subsection{Thuật toán SMOTE (Synthetic Minority Over-sampling Technique)}

SMOTE được sử dụng để xử lý mất cân bằng dữ liệu bằng cách tạo ra các mẫu tổng hợp cho lớp thiểu số \cite{ref13}.

\textbf{Giả mã thuật toán SMOTE}:

\begin{verbatim}
Algorithm: SMOTE(X_minority, N, k)
Input:
  - X_minority: Tập mẫu lớp thiểu số
  - N: Tỷ lệ oversampling (ví dụ: 200 = tăng gấp đôi)
  - k: Số lân cận gần nhất (thường k=5)
Output:
  - X_synthetic: Tập mẫu tổng hợp

1: FOR mỗi mẫu x_i in X_minority:
2:   Tìm k nearest neighbors của x_i (sử dụng Euclidean distance)
3:   FOR j = 1 TO (N/100):
4:     Chọn ngẫu nhiên một neighbor x_n từ k neighbors
5:     Tạo mẫu tổng hợp:
        x_new = x_i + rand(0,1) * (x_n - x_i)
6:     Thêm x_new vào X_synthetic
7:   END FOR
8: END FOR
9: RETURN X_minority U X_synthetic
\end{verbatim}

\textbf{Ứng dụng trong hệ thống}:
\begin{itemize}
    \item Stage 1: Tăng số lượng mẫu Normal từ ~7,769 lên khoảng 2 triệu (sampling\_strategy=0.1)
    \item Stage 2: Tăng số lượng mẫu Reconnaissance (lớp thiểu số) lên 10\% của lớp đa số
\end{itemize}

\subsection{Thuật toán tính Source Diversity Features}

Đây là thuật toán quan trọng nhất để phân biệt giữa DDoS (nhiều nguồn) và DoS (ít nguồn).

\textbf{Giả mã thuật toán}:

\begin{verbatim}
Algorithm: CalculateSourceDiversity(flows, window_size)
Input:
  - flows: Tập luồng mạng với (stime, saddr, daddr)
  - window_size: Kích thước cửa sổ thời gian (seconds)
Output:
  - diversity_features: (unique_src_count, src_entropy, top_src_ratio)

1: Tạo time_window = floor(stime / window_size)

2: FOR mỗi (time_window, target_daddr):
3:   Group = flows WHERE time_window=tw AND daddr=target_daddr
4:
5:   // Feature 1: Unique source count
6:   unique_src_count = |{saddr in Group}|
7:
8:   // Feature 2: Source entropy
9:   src_counts = frequency(saddr) FOR saddr in Group
10:  src_probs = src_counts / SUM(src_counts)
11:  src_entropy = -SUM(p_i * log2(p_i)) FOR p_i in src_probs
12:
13:  // Feature 3: Top source ratio
14:  top_src_count = max(src_counts)
15:  top_src_ratio = top_src_count / |Group|
16:
17:  ASSIGN diversity_features TO all flows IN Group
18: END FOR

19: RETURN flows WITH diversity_features
\end{verbatim}

\textbf{Ý nghĩa các features}:
\begin{itemize}
    \item \texttt{unique\_src\_count}: DDoS có giá trị cao (hàng nghìn nguồn), DoS có giá trị thấp (< 10 nguồn)
    \item \texttt{src\_entropy}: DDoS có entropy cao (phân phối đều), DoS có entropy thấp (tập trung)
    \item \texttt{top\_src\_ratio}: DDoS có tỷ lệ thấp (< 0.1), DoS có tỷ lệ cao (> 0.8)
\end{itemize}

\subsection{Mô-đun GPU Optimization}

Để tăng tốc độ huấn luyện, hệ thống sử dụng GPU (nếu có) thông qua XGBoost GPU support \cite{ref11, ref24}.

\textbf{Thuật toán phát hiện và cấu hình GPU}:

\begin{verbatim}
Algorithm: ConfigureGPU()
Output: (USE_GPU, DEVICE, TREE_METHOD)

1: TRY:
2:   Chạy lệnh: nvidia-smi --query-gpu=name,memory.total
3:   IF lệnh thành công:
4:     gpu_name, gpu_memory = parse output
5:     PRINT "GPU detected: {gpu_name}, {gpu_memory}"
6:     USE_GPU = True
7:     DEVICE = 'cuda'
8:     TREE_METHOD = 'hist'  // XGBoost 3.x uses 'hist' with device='cuda'
9:   ELSE:
10:    USE_GPU = False
11:    DEVICE = 'cpu'
12:    TREE_METHOD = 'hist'
13: EXCEPT:
14:   USE_GPU = False
15:   DEVICE = 'cpu'
16:   TREE_METHOD = 'hist'
17: RETURN (USE_GPU, DEVICE, TREE_METHOD)
\end{verbatim}

\section{Quy trình hoạt động chi tiết}

\subsection{Tổng quan quy trình}

Hệ thống hoạt động theo 8 bước chính, từ tiền xử lý dữ liệu đến đánh giá kết quả cuối cùng.

\subsection{Step 0: Merge và phân tích batches}

\textbf{Mục tiêu}: Giảm số lần đọc file và chọn batch tốt nhất cho huấn luyện.

\textbf{Input}:
\begin{itemize}
    \item 74 file CSV gốc (khoảng 230MB mỗi file)
    \item Tổng dung lượng: ~16GB
\end{itemize}

\textbf{Quy trình}:
\begin{enumerate}
    \item Chia 74 files thành 8 groups (10 files/group, group cuối 4 files)
    \item Merge mỗi group thành 1 batch file
    \item Phân tích thống kê cho mỗi batch:
    \begin{itemize}
        \item Total records
        \item Category distribution (Normal, DDoS, DoS, Reconnaissance, Theft)
        \item Missing values
        \item Protocol distribution
    \end{itemize}
    \item Lưu thống kê vào \texttt{batch\_statistics.json}
\end{enumerate}

\textbf{Output}:
\begin{itemize}
    \item 8 batch files: batch\_01.csv đến batch\_08.csv (~2-2.6GB mỗi file)
    \item File thống kê: batch\_statistics.json
\end{itemize}

\textbf{Lựa chọn batch cho training}: Chọn batch\_01 và batch\_04 vì có số lượng Normal samples cao nhất (cần thiết cho Stage 1).

\subsection{Step 1: Tạo balanced test set}

\textbf{Mục tiêu}: Tạo tập test cân bằng để đánh giá công bằng.

\textbf{Input}: batch\_02.csv (chọn làm nguồn test vì không dùng cho training)

\textbf{Quy trình}:
\begin{enumerate}
    \item Sample từ batch\_02:
    \begin{itemize}
        \item Normal: 2,000 samples
        \item DDoS: 35,000 samples
        \item DoS: 50,000 samples
        \item Reconnaissance: 13,000 samples
    \end{itemize}
    \item Shuffle và lưu thành \texttt{test\_balanced\_100k.csv}
\end{enumerate}

\textbf{Output}: Tập test cân bằng với 100,000 records.

\subsection{Step 2: Load training data}

\textbf{Input}: batch\_01.csv (~10M records) và batch\_04.csv (~10M records).

\textbf{Quy trình}:
\begin{enumerate}
    \item Đọc batch\_01.csv với \texttt{pd.read\_csv(low\_memory=False)}
    \item Đọc batch\_04.csv
    \item Concatenate 2 DataFrames: \texttt{df\_train = pd.concat([df1, df2])}
    \item Garbage collection: \texttt{del df1, df2; gc.collect()}
\end{enumerate}

\textbf{Output}: \texttt{df\_train} với 20 triệu records.

\subsection{Step 3: Load test data}

\textbf{Input}: test\_balanced\_100k.csv

\textbf{Output}: \texttt{df\_test} với 100,000 records (cân bằng).

\subsection{Step 4: Feature Engineering}

Đây là bước quan trọng nhất, quyết định chất lượng của mô hình.

\textbf{Bước 4.1: Tạo Source Diversity Features}

Input: \texttt{df\_train} và \texttt{df\_test}.

Quy trình:
\begin{enumerate}
    \item Tạo time\_window = floor(stime / 30) // Window 30 giây
    \item Group by (time\_window, daddr)
    \item Với mỗi group:
    \begin{itemize}
        \item Tính unique\_src\_count = số lượng saddr duy nhất
        \item Tính src\_entropy = $-\sum(p_i \log_2 p_i)$
        \item Tính top\_src\_ratio = max\_count / total\_count
    \end{itemize}
    \item Merge features trở lại \texttt{df\_train} và \texttt{df\_test}
    \item Fill NaN với giá trị mặc định (1, 0.0, 1.0)
\end{enumerate}

\textbf{Quan trọng}: Train và test PHẢI xử lý RIÊNG BIỆT để tránh data leakage.

\textbf{Bước 4.2: Feature Selection}

Columns cần loại bỏ: Identifiers (pkSeqID, saddr, etc.), MAC addresses, Labels, Timestamps.

Features cuối cùng (22 features):
\begin{itemize}
    \item Basic: flgs, proto, pkts, bytes, state, seq, dur, spkts, dpkts, sbytes, dbytes
    \item Statistical: mean, stddev, sum, min, max, rate, srate, drate
    \item \textbf{Diversity (New)}: unique\_src\_count, src\_entropy, top\_src\_ratio
\end{itemize}

\textbf{Bước 4.3: Xử lý Missing Values}

Tính median từ TRAIN set và fill NaN cho cả TRAIN và TEST set (tránh leakage).

\textbf{Bước 4.4: Encoding Categorical Features}

Fit LabelEncoder chỉ trên TRAIN data, sau đó transform TEST data (handle unknown values).

\subsection{Step 5: Train Stage 1 (Binary Classification)}

\textbf{Mục tiêu}: Phân biệt Attack vs Normal.

\textbf{Quy trình}:
\begin{enumerate}
    \item Tạo binary labels (Normal = 0, Attack = 1).
    \item Stratified Split (85\%-15\%).
    \item SMOTE Oversampling (tăng Normal lên 10\% của Attack).
    \item Huấn luyện XGBoost với \texttt{scale\_pos\_weight} và GPU support.
\end{enumerate}

\textbf{Kết quả Stage 1}: Accuracy 99.26\%, ROC-AUC 99.99\%.

\subsection{Step 6: Train Stage 2 (Multi-class Classification)}

\textbf{Mục tiêu}: Phân loại DDoS vs DoS vs Reconnaissance.

\textbf{Quy trình}:
\begin{enumerate}
    \item Filter chỉ lấy attack samples.
    \item Map labels (DDoS=0, DoS=1, Recon=2).
    \item Stratified Split.
    \item Sử dụng \texttt{sample\_weight='balanced'} (thay vì SMOTE quá mức).
    \item Huấn luyện XGBoost Multi-class.
\end{enumerate}

\textbf{Kết quả Stage 2}: Accuracy 97.58\%.

\subsection{Step 7: Combined Pipeline Evaluation}

Kết hợp dự đoán Stage 1 và Stage 2. Overall Accuracy: 97.19\%.

\section{Công cụ, công nghệ và nền tảng triển khai}

\subsection{Ngôn ngữ lập trình và thư viện}

\textbf{Ngôn ngữ}: Python 3.10.

\textbf{Thư viện chính}:
\begin{itemize}
    \item Pandas, NumPy: Xử lý dữ liệu.
    \item XGBoost: Model training.
    \item Scikit-learn, Imbalanced-learn: Preprocessing và metrics.
\end{itemize}

\subsection{Môi trường huấn luyện}

\textbf{Nền tảng}: Google Colab Pro+.
\begin{itemize}
    \item RAM: 52 GB.
    \item GPU: Tesla T4 (16GB VRAM) / V100 / A100.
\end{itemize}

\section{Hệ thống phân loại chi tiết các biến thể DDoS}

\subsection{Động lực và mục tiêu}

Các cuộc tấn công DDoS trong môi trường IoT ngày càng đa dạng về cơ chế và mục tiêu, bao gồm tấn công băng thông, tấn công tài nguyên kết nối, và tấn công lớp ứng dụng. Trong Bot-IoT dataset \cite{ref1}, các tấn công DDoS được gán nhãn chi tiết theo trường \texttt{subcategory}, tương ứng với ba biến thể chính:

\textbf{DDoS-HTTP} -- Tấn công lớp ứng dụng (Layer 7), khai thác số lượng lớn HTTP requests nhằm làm quá tải web server. Loại tấn công này thường được thực thi bởi các công cụ như GoldenEye \cite{ref27}.

\textbf{DDoS-TCP} -- Tấn công lớp vận chuyển (Transport layer), bao gồm SYN flood và ACK flood, đánh vào bảng trạng thái kết nối của thiết bị mạng. Các biến thể này thường được tạo bởi công cụ Hping3 \cite{ref28}.

\textbf{DDoS-UDP} -- Tấn công băng thông bằng cách gửi lượng lớn UDP packets đến các cổng ngẫu nhiên, gây tiêu tốn tài nguyên mạng và làm nghẽn dịch vụ.

Báo cáo thống kê từ Kaspersky Q3 2021 \cite{ref26} cho thấy SYN flooding và UDP flooding chiếm tỷ lệ cao trong các cuộc tấn công thực tế. Điều này khẳng định tầm quan trọng của hệ thống phân loại chi tiết nhằm:

\begin{itemize}
    \item Nhận diện chính xác các đặc trưng hành vi của từng loại DDoS
    \item Hỗ trợ lựa chọn biện pháp phòng thủ phù hợp, chẳng hạn rate limiting cho HTTP floods, SYN cookies cho TCP floods, hoặc packet filtering cho UDP floods
    \item Phân tích forensics và threat intelligence, giúp suy luận công cụ tấn công, chiến thuật botnet và dấu hiệu hành vi (behavioral signatures)
\end{itemize}

\subsection{Dataset và phương pháp cân bằng}

Bot-IoT dataset là một trong những bộ dữ liệu IoT-IDS quy mô lớn nhất hiện nay. Để xây dựng mô hình phân loại chi tiết các biến thể DDoS, nghiên cứu tập trung vào ba loại tấn công DDoS chính: DDoS-HTTP, DDoS-TCP và DDoS-UDP. Việc xử lý mất cân bằng giữa các biến thể DDoS trở thành yếu tố then chốt để đảm bảo mô hình không thiên lệch.

\subsubsection{Hạn chế của SMOTE trong bài toán DDoS IoT}

Các kỹ thuật oversampling tổng hợp như SMOTE thường không phù hợp với lưu lượng DDoS vì:

\textbf{Mất tương quan thời gian (temporal correlation)}:
DDoS thể hiện hành vi bursty và biểu hiện dạng chuỗi theo thời gian. SMOTE tạo mẫu tổng hợp rời rạc trong feature space, gây phá vỡ cấu trúc chuỗi của các flows tấn công.

\textbf{Không phản ánh đúng signatures của botnet}:
Các packet trains trong tấn công thực tế có đặc trưng riêng theo tool và topology botnet, không thể nội suy tuyến tính từ vài điểm dữ liệu.

\textbf{Nguy cơ overfitting}:
Model có thể học đặc trưng nhân tạo sinh ra bởi SMOTE thay vì đặc trưng thực tế của hành vi tấn công.

\subsubsection{Random Consecutive Sampling}

Thay vì SMOTE, nghiên cứu áp dụng Random Consecutive Sampling nhằm:

\begin{itemize}
    \item \textbf{Bảo toàn hành vi theo thời gian}, bằng cách lấy các cụm flows liên tiếp từ mỗi subcategory
    \item \textbf{Tránh tạo dữ liệu giả}, giữ nguyên bản chất distribution của tấn công
    \item \textbf{Đảm bảo cân bằng}, với khoảng 7,634--7,635 mẫu cho mỗi biến thể DDoS: DDoS-HTTP, DDoS-TCP, DDoS-UDP
\end{itemize}

\textbf{Lựa chọn kích thước mẫu}: Số lượng 7,634--7,635 samples được chọn để đảm bảo tính cân bằng hoàn toàn giữa ba biến thể DDoS. Sau quá trình tiền xử lý (data cleaning, loại bỏ outliers, xử lý missing values), các attack subcategories (DDoS-HTTP, DDoS-TCP, DDoS-UDP) đều được down-sampling về khoảng 7,634--7,635 mẫu bằng Random Consecutive Sampling. Phương pháp này đảm bảo:

\begin{itemize}
    \item Perfect class balance (~33.3\% cho mỗi biến thể DDoS)
    \item Không cần class weights trong XGBoost
    \item Model học công bằng từ tất cả các DDoS variants
\end{itemize}

Bảng \ref{tab:ddos_distribution} trình bày phân phối cân bằng thu được:

\begin{table}[H]
    \centering
    \caption{Phân phối cân bằng các biến thể DDoS}
    \label{tab:ddos_distribution}
    \begin{tabular}{|l|r|r|}
    \hline
    \textbf{DDoS Variant} & \textbf{Samples} & \textbf{Tỷ lệ} \\ \hline
    DDoS-HTTP & 7,635 & 33.3\% \\ \hline
    DDoS-TCP & 7,635 & 33.3\% \\ \hline
    DDoS-UDP & 7,634 & 33.3\% \\ \hline
    \textbf{Tổng} & 22,904 & 100\% \\ \hline
    \end{tabular}
\end{table}

Phương pháp này giúp mô hình không bị ảnh hưởng bởi bias phân phối, đồng thời vẫn giữ được cấu trúc thực tế của traffic.

\subsection{Kiến trúc mô hình phân loại đa lớp}

Hệ thống sử dụng mô hình XGBoost Multi-class Classifier chuyên biệt cho phân loại các biến thể DDoS với các đặc điểm:

\begin{itemize}
    \item \textbf{Input}: 16 features đã chọn lọc
    \item \textbf{Output}: 3 lớp (DDoS-HTTP, DDoS-TCP, DDoS-UDP)
    \item \textbf{Không sử dụng SMOTE}, nhờ dataset đã được cân bằng
    \item \textbf{Ưu điểm}: tốc độ huấn luyện nhanh, khả năng xử lý tabular data tốt, phù hợp môi trường IDS theo thời gian thực
\end{itemize}

\subsection{Lựa chọn đặc trưng -- 16 Best Features}

Từ 35 features gốc của Bot-IoT flow records, 16 đặc trưng được chọn dựa trên tính phân biệt (discriminative power) và mức độ liên quan hành vi (behavioral relevance). Các nhóm feature chính gồm:

\subsubsection{(1) Basic flow metrics (4 features)}

\texttt{pkts}, \texttt{bytes} -- phản ánh lưu lượng tấn công

\texttt{seq} -- phát hiện bất thường trong TCP sequence patterns

\texttt{dur} -- độ dài flow, phân biệt sustained HTTP floods và short-burst TCP floods

\subsubsection{(2) Statistical aggregations (5 features)}

\texttt{mean}, \texttt{stddev}, \texttt{sum}, \texttt{min}, \texttt{max} của inter-arrival/duration

$\rightarrow$ ghi nhận burstiness và sự biến thiên trong traffic

\subsubsection{(3) Directional metrics (4 features)}

\texttt{spkts}, \texttt{dpkts}, \texttt{sbytes}, \texttt{dbytes}

$\rightarrow$ DDoS thường biểu hiện bất cân xứng mạnh (nhiều nguồn $\rightarrow$ một đích)

\subsubsection{(4) Rate-based features (3 features)}

\texttt{rate}, \texttt{srate}, \texttt{drate}

$\rightarrow$ phản ánh intensity của cuộc tấn công

\subsubsection{Lý do loại bỏ các nhóm feature khác}

\textbf{Timestamps} (\texttt{stime}, \texttt{ltime}):
Lưu lượng tấn công có thể đến bất kỳ thời điểm nào; dùng timestamps dễ khiến mô hình ``học thuộc'' thời gian và không generalize khi triển khai online.

\textbf{Địa chỉ IP/MAC}:
\begin{itemize}
    \item Cardinality rất lớn
    \item Botnets thường thay đổi IP để né tránh
    \item Không có giá trị trong hệ thống IDS tổng quát
\end{itemize}

\textbf{Protocol-specific metadata} (\texttt{proto}, \texttt{flags}, \texttt{state}):
Dù hữu ích trong một số trường hợp, nhưng mô hình phân loại variants DDoS cần ưu tiên behavioral features, không phụ thuộc vào header-level semantics, nhằm tăng khả năng phát hiện cross-protocol.

\subsection{Siêu tham số}

\begin{table}[H]
    \centering
    \caption{Cấu hình XGBoost DDoS Classifier}
    \begin{tabular}{|l|l|}
    \hline
    \textbf{Tham số} & \textbf{Giá trị} \\ \hline
    objective & multi:softmax \\ \hline
    num\_class & 4 \\ \hline
    max\_depth & 7 \\ \hline
    learning\_rate & 0.1 \\ \hline
    n\_estimators & 100 \\ \hline
    subsample & 0.8 \\ \hline
    colsample\_bytree & 0.8 \\ \hline
    eval\_metric & mlogloss \\ \hline
    \end{tabular}
\end{table}

\subsection{Quy trình huấn luyện}

\textbf{Bước 1}: Load dữ liệu từ \texttt{balanced\_ddos\_only.csv}, convert to numeric, xử lý NaN/inf

\textbf{Bước 2}: Chọn 16 features, label encoding (HTTP=0, TCP=1, UDP=2), standardization

\textbf{Bước 3}: Split 80/10/10 với stratification

\textbf{Bước 4}: Train XGBoost với eval\_set, converge trong 20-30 epochs, $<$ 2 phút trên CPU

\textbf{Bước 5}: Evaluate với Accuracy/Precision/Recall/F1, tạo confusion matrix, save model

\section{Phân tích độ phức tạp và đánh giá tối ưu hóa}

\subsection{Độ phức tạp về thời gian (Time Complexity)}

\textbf{XGBoost Training}:
Với GPU (parallelization), thời gian huấn luyện giảm đáng kể:
\begin{itemize}
    \item Speedup: 100-200x so với CPU.
    \item Thời gian thực tế: ~30 giây (Stage 1) + ~59 giây (Stage 2) = ~1.5 phút.
\end{itemize}

\textbf{Inference (Combined Pipeline)}:
Sử dụng vectorization giúp giảm thời gian dự đoán cho 100k samples xuống còn ~2 giây.

\subsection{Độ phức tạp về không gian (Space Complexity)}

Peak RAM usage đo được là 33GB, nằm trong giới hạn 52GB của Colab Pro+.

\subsection{Đánh giá tối ưu hóa}

Hệ thống đã đạt được các mục tiêu tối ưu hóa:
\begin{itemize}
    \item \textbf{Training Time}: 12 phút (mục tiêu 15-20 phút).
    \item \textbf{RAM Usage}: 33GB (mục tiêu < 52GB).
    \item \textbf{Accuracy}: 97.19\%.
    \item \textbf{Scalability}: Có thể mở rộng lên 5x dữ liệu.
\end{itemize}

\chapter{THỰC NGHIỆM VÀ THẢO LUẬN}

\section{Môi trường thực nghiệm}

\subsection{Cấu hình phần cứng}

Hệ thống được huấn luyện và đánh giá trên nền tảng Google Colab Pro+ với cấu hình chi tiết như sau:

\begin{table}[H]
    \centering
    \caption{Cấu hình phần cứng}
    \begin{tabular}{|l|l|l|}
    \hline
    \textbf{Thành phần} & \textbf{Cấu hình} & \textbf{Ghi chú} \\ \hline
    \textbf{Nền tảng} & Google Colab Pro+ & Cloud computing platform \\ \hline
    \textbf{CPU} & Intel Xeon @ 2.0-2.3GHz & 2 cores allocated \\ \hline
    \textbf{RAM} & 52 GB High-RAM & Peak usage: 33 GB \\ \hline
    \textbf{GPU} & Tesla T4 (16GB VRAM) & CUDA 11.x, cuDNN 8.x \\ \hline
    \textbf{Disk} & 100+ GB SSD & Google Drive mount \\ \hline
    \textbf{Network} & Google Cloud network & \textasciitilde100 Mbps download \\ \hline
    \textbf{Session timeout} & 12 hours & Auto-disconnect after idle \\ \hline
    \end{tabular}
\end{table}

\textbf{Lý do lựa chọn}:
\begin{itemize}
    \item Chi phí hợp lý (~\$50/tháng cho Colab Pro+) so với việc đầu tư phần cứng chuyên dụng.
    \item GPU mạnh mẽ (T4) giúp giảm thời gian huấn luyện từ 4 giờ xuống 12 phút.
    \item RAM 52GB đủ để xử lý 20 triệu records.
    \item Phù hợp với môi trường giáo dục và nghiên cứu.
\end{itemize}

\subsection{Môi trường phần mềm}

\begin{table}[H]
    \centering
    \caption{Thư viện và phiên bản}
    \begin{tabular}{|l|l|l|}
    \hline
    \textbf{Thư viện} & \textbf{Phiên bản} & \textbf{Vai trò} \\ \hline
    \textbf{Python} & 3.10.12 & Ngôn ngữ lập trình chính \\ \hline
    \textbf{XGBoost} & 3.0.0 & Mô hình gradient boosting với GPU support \\ \hline
    \textbf{scikit-learn} & 1.3.2 & Preprocessing, metrics, validation \\ \hline
    \textbf{imbalanced-learn} & 0.11.0 & SMOTE oversampling \\ \hline
    \textbf{pandas} & 2.1.3 & DataFrame operations \\ \hline
    \textbf{numpy} & 1.24.3 & Numerical computations \\ \hline
    \textbf{scipy} & 1.11.4 & Entropy calculation \\ \hline
    \textbf{matplotlib} & 3.8.0 & Visualization \\ \hline
    \textbf{seaborn} & 0.13.0 & Statistical plots \\ \hline
    \textbf{psutil} & 5.9.6 & System monitoring \\ \hline
    \textbf{joblib} & 1.3.2 & Model serialization \\ \hline
    \end{tabular}
\end{table}

\textbf{Hệ điều hành}: Ubuntu 22.04 LTS (kernel 5.15).
\textbf{CUDA Toolkit}: CUDA 11.8, cuDNN 8.6 (cho GPU acceleration).

\subsection{Tập dữ liệu}

\textbf{Dataset}: Bot-IoT Dataset được công bố bởi UNSW Canberra Cyber (2019).

\textbf{Đặc điểm tập dữ liệu}:

\begin{table}[H]
    \centering
    \caption{Thống kê tập dữ liệu Bot-IoT}
    \begin{tabular}{|l|l|l|}
    \hline
    \textbf{Đặc điểm} & \textbf{Giá trị} & \textbf{Mô tả} \\ \hline
    \textbf{Số files} & 74 files CSV & Raw network flow data \\ \hline
    \textbf{Tổng dung lượng} & \textasciitilde16 GB & Uncompressed \\ \hline
    \textbf{Tổng số records} & 40-50 triệu & Network flows \\ \hline
    \textbf{Số features} & 35 features & Gốc (trước feature engineering) \\ \hline
    \textbf{Thời gian thu thập} & 2018-2019 & Real IoT testbed \\ \hline
    \textbf{Loại attacks} & 4 categories & DDoS, DoS, Reconnaissance, Theft \\ \hline
    \textbf{Imbalance ratio} & 2000:1 & Attack:Normal \\ \hline
    \end{tabular}
\end{table}

\textbf{Phân phối lớp trong toàn bộ dataset}:

\begin{table}[H]
    \centering
    \caption{Phân phối lớp (Full Dataset)}
    \begin{tabular}{|l|l|l|}
    \hline
    \textbf{Category} & \textbf{Số lượng} & \textbf{Tỷ lệ (\%)} \\ \hline
    \textbf{DoS} & \textasciitilde26 triệu & 52\% \\ \hline
    \textbf{DDoS} & \textasciitilde10 triệu & 20\% \\ \hline
    \textbf{Reconnaissance} & \textasciitilde3.6 triệu & 7.2\% \\ \hline
    \textbf{Normal} & \textasciitilde20,000 & 0.04\% \\ \hline
    \textbf{Theft} & \textasciitilde3,000 & 0.006\% \\ \hline
    \textbf{Tổng} & \textasciitilde50 triệu & 100\% \\ \hline
    \end{tabular}
\end{table}

\textbf{Training set} (sau khi chọn batch\_01 + batch\_04):
\begin{itemize}
    \item Total: 20,000,000 records
    \item Normal: 7,769 (0.039\%)
    \item Attack: 19,992,231 (99.961\%)
    \begin{itemize}
        \item DoS: 13,005,877 (65\%)
        \item DDoS: 5,163,128 (26\%)
        \item Reconnaissance: 1,821,639 (9\%)
        \item Theft: 1,587 (loại bỏ vì quá ít)
    \end{itemize}
\end{itemize}

\textbf{Test set} (balanced):
\begin{itemize}
    \item Total: 100,000 records
    \item Normal: 2,000 (2\%)
    \item DDoS: 35,000 (35\%)
    \item DoS: 50,000 (50\%)
    \item Reconnaissance: 13,000 (13\%)
\end{itemize}

\textbf{Lý do tạo balanced test set}: Tập test gốc có tỷ lệ mất cân bằng nghiêm trọng, khiến accuracy không phản ánh đúng khả năng phát hiện Normal. Tập test cân bằng cho phép đánh giá công bằng hiệu năng trên tất cả các lớp.

\begin{figure}[H]
    \centering
    \includegraphics[width=0.8\textwidth]{figures/05_class_distribution.png}
    \caption{Phân phối lớp trong tập dữ liệu - So sánh giữa training set (imbalanced) và balanced test set.}
    \label{fig:class_dist}
\end{figure}

\subsection{Siêu tham số mô hình}

\textbf{Stage 1: Binary Classification (Attack vs Normal)}

\begin{table}[H]
    \centering
    \caption{Siêu tham số Stage 1}
    \begin{tabular}{|l|l|l|}
    \hline
    \textbf{Tham số} & \textbf{Giá trị} & \textbf{Ý nghĩa} \\ \hline
    n\_estimators & 200 & Số lượng cây quyết định \\ \hline
    max\_depth & 6 & Độ sâu tối đa của mỗi cây \\ \hline
    learning\_rate & 0.1 & Tốc độ học (shrinkage) \\ \hline
    subsample & 0.8 & Tỷ lệ sample cho mỗi cây (80\%) \\ \hline
    colsample\_bytree & 0.8 & Tỷ lệ features cho mỗi cây (80\%) \\ \hline
    scale\_pos\_weight & 2574 & Trọng số cho lớp thiểu số (Normal) \\ \hline
    tree\_method & 'hist' & Histogram-based algorithm \\ \hline
    device & 'cuda' & Sử dụng GPU \\ \hline
    eval\_metric & 'logloss' & Binary cross-entropy \\ \hline
    early\_stopping\_rounds & 20 & Dừng sớm nếu không cải thiện \\ \hline
    \end{tabular}
\end{table}

\textbf{SMOTE cho Stage 1}:
\begin{itemize}
    \item sampling\_strategy: 0.1 (tăng Normal lên 10\% của Attack)
    \item k\_neighbors: 5
\end{itemize}

\textbf{Stage 2: Multi-class Classification (DDoS vs DoS vs Reconnaissance)}

\begin{table}[H]
    \centering
    \caption{Siêu tham số Stage 2}
    \begin{tabular}{|l|l|l|}
    \hline
    \textbf{Tham số} & \textbf{Giá trị} & \textbf{Ý nghĩa} \\ \hline
    objective & 'multi:softmax' & Multi-class classification \\ \hline
    num\_class & 3 & Số lớp (DDoS, DoS, Reconnaissance) \\ \hline
    n\_estimators & 200 & Số lượng cây \\ \hline
    max\_depth & 6 & Độ sâu tối đa \\ \hline
    learning\_rate & 0.1 & Tốc độ học \\ \hline
    subsample & 0.8 & Tỷ lệ sample \\ \hline
    colsample\_bytree & 0.8 & Tỷ lệ features \\ \hline
    tree\_method & 'hist' & Histogram-based \\ \hline
    device & 'cuda' & GPU acceleration \\ \hline
    eval\_metric & 'mlogloss' & Multi-class cross-entropy \\ \hline
    sample\_weight & 'balanced' & Cân bằng trọng số cho các lớp \\ \hline
    \end{tabular}
\end{table}

\section{Kết quả thực nghiệm}

\subsection{Kết quả huấn luyện}

\begin{table}[H]
    \centering
    \caption{Kết quả huấn luyện tổng hợp}
    \begin{tabular}{|l|l|l|l|}
    \hline
    \textbf{Metric} & \textbf{Stage 1} & \textbf{Stage 2} & \textbf{Overall} \\ \hline
    \textbf{Accuracy} & 99.26\% & 97.58\% & 97.19\% \\ \hline
    \textbf{ROC-AUC} & 99.99\% & - & - \\ \hline
    \textbf{Training Time} & 30.6s & 58.7s & 89.3s \\ \hline
    \textbf{Best Iteration} & 199/200 & 199/200 & - \\ \hline
    \end{tabular}
\end{table}

\begin{figure}[H]
    \centering
    \includegraphics[width=0.8\textwidth]{figures/00_summary_dashboard.png}
    \caption{Overall Performance Summary Dashboard}
    \label{fig:dashboard}
\end{figure}

\subsection{Confusion Matrix và phân tích chi tiết}

\textbf{Stage 1: Binary Classification}

\begin{table}[H]
    \centering
    \caption{Confusion Matrix - Stage 1}
    \begin{tabular}{|l|l|l|}
    \hline
    & \textbf{Pred: Normal} & \textbf{Pred: Attack} \\ \hline
    \textbf{Act: Normal} & 1,999 (TN) & 1 (FP) \\ \hline
    \textbf{Act: Attack} & 744 (FN) & 97,256 (TP) \\ \hline
    \end{tabular}
\end{table}

\textbf{Metrics từ Confusion Matrix}:
\begin{itemize}
    \item True Positive Rate (Recall): 99.24\%
    \item False Positive Rate: 0.05\%
    \item True Negative Rate (Specificity): 99.95\%
\end{itemize}

\textbf{Overall Pipeline: 4x4 Confusion Matrix}

\begin{table}[H]
    \centering
    \caption{Confusion Matrix - Overall (4 categories)}
    \begin{tabular}{|l|l|l|l|l|}
    \hline
    & \textbf{Pred: Norm} & \textbf{Pred: DDoS} & \textbf{Pred: DoS} & \textbf{Pred: Recon} \\ \hline
    \textbf{Act: Norm} & 1,999 & 0 & 1 & 0 \\ \hline
    \textbf{Act: DDoS} & 513 & 32,628 & 29 & 1,830 \\ \hline
    \textbf{Act: DoS} & 166 & 15 & 49,679 & 140 \\ \hline
    \textbf{Act: Recon} & 65 & 0 & 49 & 12,886 \\ \hline
    \end{tabular}
\end{table}

\textbf{Phân tích}:
\begin{itemize}
    \item \textbf{Normal}: Recall 99.95\% - xuất sắc.
    \item \textbf{DDoS}: Recall 93.22\% - điểm yếu chính (bị nhầm với Recon và Normal).
    \item \textbf{DoS}: Recall 99.36\% - rất tốt.
    \item \textbf{Reconnaissance}: Recall 99.12\% - tốt.
\end{itemize}

\begin{figure}[H]
    \centering
    \includegraphics[width=0.7\textwidth]{figures/01_confusion_matrix.png}
    \caption{Confusion Matrix cho Overall Pipeline (4 categories)}
    \label{fig:cm}
\end{figure}

\subsection{Training Curves}

Loss giảm nhanh trong 50 iterations đầu tiên và converge ổn định sau iteration 100, không có dấu hiệu overfitting.

\begin{figure}[H]
    \centering
    \includegraphics[width=0.7\textwidth]{figures/combined_loss_curves.png}
    \caption{Combined Training Curves - So sánh loss evolution của cả hai stages}
    \label{fig:loss}
\end{figure}

\section{Đánh giá hiệu năng}

\subsection{Đánh giá chi tiết theo từng loại}

\begin{table}[H]
    \centering
    \caption{Metrics chi tiết cho từng category}
    \begin{tabular}{|l|l|l|l|l|}
    \hline
    \textbf{Category} & \textbf{Precision} & \textbf{Recall} & \textbf{F1-Score} & \textbf{Accuracy} \\ \hline
    \textbf{Normal} & 72.88\% & 99.95\% & 84.29\% & 99.95\% \\ \hline
    \textbf{DDoS} & 99.95\% & 93.22\% & 96.47\% & 93.22\% \\ \hline
    \textbf{DoS} & 99.84\% & 99.36\% & 99.60\% & 99.36\% \\ \hline
    \textbf{Reconnaissance} & 86.74\% & 99.12\% & 92.52\% & 99.12\% \\ \hline
    \textbf{Avg} & 97.35\% & 97.19\% & 97.17\% & 97.19\% \\ \hline
    \end{tabular}
\end{table}

\begin{figure}[H]
    \centering
    \includegraphics[width=0.7\textwidth]{figures/03_per_category_metrics.png}
    \caption{Precision, Recall và F1-Score cho từng category}
    \label{fig:metrics}
\end{figure}

\subsection{Đánh giá hiệu năng tính toán}

\begin{table}[H]
    \centering
    \caption{Hiệu năng tính toán}
    \begin{tabular}{|l|l|l|l|}
    \hline
    \textbf{Metric} & \textbf{Stage 1} & \textbf{Stage 2} & \textbf{Overall} \\ \hline
    \textbf{Throughput (samples/sec)} & 1,017,595 & 2,908,013 & 1,217,890 \\ \hline
    \textbf{Latency per sample} & 0.98 $\mu$s & 0.34 $\mu$s & 0.82 $\mu$s \\ \hline
    \textbf{Model Size} & 2.1 MB & 6.3 MB & 8.4 MB \\ \hline
    \end{tabular}
\end{table}


\section{Kết quả phân loại chi tiết các biến thể DDoS}

Phần này trình bày hiệu năng của mô hình phân loại các biến thể DDoS (DDoS-HTTP, DDoS-TCP, DDoS-UDP) dựa trên tập dữ liệu đã cân bằng và trích xuất 16 đặc trưng hành vi mạng. Các kết quả được minh họa qua phân bố lớp, tương quan đặc trưng, quá trình huấn luyện, metrics tổng thể và đánh giá chi tiết theo từng biến thể.

\subsection{Phân bố các lớp sau khi cân bằng}

Hình dưới đây cho thấy phân bố ba biến thể DDoS trong tập dữ liệu sau khi áp dụng phương pháp Random Consecutive Sampling.

\begin{figure}[H]
    \centering
    \includegraphics[width=0.7\textwidth]{plots/class_distribution.png}
    \caption{Phân bố các biến thể DDoS sau Random Consecutive Sampling}
    \label{fig:ddos_class_distribution}
\end{figure}

\textbf{Nhận xét}:
\begin{itemize}
    \item Mỗi biến thể DDoS có 7,634--7,635 mẫu, thể hiện phân bố gần như đồng đều (~33.3\% mỗi lớp)
    \item Điều này ngăn mô hình bị bias theo một loại tấn công DDoS cụ thể
    \item Với phân bố cân bằng, mô hình có điều kiện học representation cho từng biến thể DDoS một cách công bằng
\end{itemize}

\subsection{Phân tích tương quan giữa 16 đặc trưng}

\begin{figure}[H]
    \centering
    \includegraphics[width=0.85\textwidth]{plots/correlation_matrix.png}
    \caption{Feature Correlation Matrix của 16 đặc trưng}
    \label{fig:ddos_correlation_matrix}
\end{figure}

\textbf{Nhận xét từ heatmap}:
\begin{itemize}
    \item \textbf{Nhóm pkts, bytes, spkts, sbytes, dpkts, dbytes} có tương quan rất cao ($\geq$0.90) \\
    $\rightarrow$ Đây là các chỉ số phản ánh cường độ traffic, giúp phân biệt rõ UDP/TCP floods
    
    \item \textbf{Các đặc trưng thời gian} như mean, stddev, sum có tương quan mạnh với dur \\
    $\rightarrow$ Hỗ trợ nhận diện HTTP flood vốn có duration dài hơn
    
    \item \textbf{Một số đặc trưng gần như không tương quan} ($\approx$0) \\
    $\rightarrow$ Cho thấy 16 đặc trưng chọn lọc có sự đa dạng, không trùng lặp thông tin
\end{itemize}

\textbf{Kết luận}: Heatmap chứng minh bộ 16 features có tính phân biệt cao và phù hợp cho phân loại đa biến thể DDoS.

\subsection{Quá trình huấn luyện mô hình}

\begin{figure}[H]
    \centering
    \includegraphics[width=0.75\textwidth]{plots/training_history.png}
    \caption{Training History - Loss curves theo epochs}
    \label{fig:ddos_training_history}
\end{figure}

\textbf{Phân tích}:
\begin{itemize}
    \item Train loss và validation loss gần như trùng nhau $\rightarrow$ không có overfitting
    \item Loss giảm đều và nhanh, đặc biệt trong 20 epoch đầu $\rightarrow$ mô hình học rất hiệu quả
    \item Sau $\sim$40 epoch, loss gần như hội tụ $\rightarrow$ đảm bảo tính ổn định cho mô hình phân loại
\end{itemize}

\textbf{Nhận định}: Bộ đặc trưng và kiến trúc mô hình là hoàn toàn phù hợp, không gây nhiễu hoặc khó học.

\subsection{Kết quả đánh giá tổng thể}

\begin{figure}[H]
    \centering
    \includegraphics[width=0.8\textwidth]{plots/overall_metrics.png}
    \caption{Overall Model Performance Metrics}
    \label{fig:ddos_overall_metrics}
\end{figure}

\textbf{Kết quả}:

\begin{table}[H]
    \centering
    \begin{tabular}{|l|c|}
    \hline
    \textbf{Metric} & \textbf{Giá trị} \\ \hline
    Accuracy & 99.41\% \\ \hline
    Precision & 99.48\% \\ \hline
    Recall & 99.36\% \\ \hline
    F1-Score & 99.42\% \\ \hline
    \end{tabular}
\end{table}

$\rightarrow$ Tất cả các metrics đều đạt trên 99\%, thể hiện hiệu năng rất cao.

$\rightarrow$ Cho thấy mô hình phân loại nhất quán, cân bằng giữa precision và recall.

\subsection{Ma trận nhầm lẫn (Confusion Matrix)}

\begin{figure}[H]
    \centering
    \includegraphics[width=0.75\textwidth]{plots/confusion_matrix.png}
    \caption{Confusion Matrix for DDoS Variants Classification}
    \label{fig:ddos_confusion_matrix}
\end{figure}

\textbf{Nhận xét chi tiết}:
\begin{itemize}
    \item \textbf{HTTP} $\rightarrow$ 948/954 đúng (99.37\%), có 6 mẫu bị nhầm sang TCP
    \item \textbf{TCP} $\rightarrow$ 948/955 đúng (99.27\%), có 7 mẫu bị nhầm sang HTTP
    \item \textbf{UDP} $\rightarrow$ 954/954 đúng (100\%), phân loại hoàn hảo
\end{itemize}

$\rightarrow$ Đây là mức độ chính xác rất cao cho bài toán phân loại DDoS variants.

\textbf{Điểm đáng chú ý}: Có một số nhầm lẫn nhỏ giữa HTTP và TCP (tổng 13 mẫu), có thể do HTTP floods và TCP floods đều hoạt động ở transport/application layer và có một số đặc trưng chồng lấn.

\subsection{Hiệu năng theo từng lớp}

\begin{figure}[H]
    \centering
    \includegraphics[width=0.8\textwidth]{plots/per_class_metrics.png}
    \caption{Per-Class Performance Metrics}
    \label{fig:ddos_per_class_metrics}
\end{figure}

\textbf{Phân tích theo từng nhóm}:

\textbf{Precision}:
\begin{itemize}
    \item HTTP: 99.3\%
    \item TCP: 99.2\%
    \item UDP: 100\%
\end{itemize}
$\rightarrow$ UDP đạt precision hoàn hảo, HTTP và TCP có precision rất cao (~99\%).

\textbf{Recall}:
\begin{itemize}
    \item HTTP: 99.4\%
    \item TCP: 99.3\%
    \item UDP: 100\%
\end{itemize}
$\rightarrow$ UDP đạt recall hoàn hảo, HTTP và TCP có recall rất cao (~99\%).

\textbf{F1-score}:
\begin{itemize}
    \item HTTP: 99.35\%
    \item TCP: 99.25\%
    \item UDP: 100\%
\end{itemize}
$\rightarrow$ Cân bằng tốt giữa precision và recall cho tất cả các biến thể DDoS.

\textbf{Support}:
\begin{itemize}
    \item Mỗi lớp test gồm 954--955 mẫu, đảm bảo tính khách quan của đánh giá
\end{itemize}

\section{Phân tích và thảo luận kết quả}

\subsection{Điểm thành công}

\textbf{Two-Stage Hierarchical Model}:
\begin{enumerate}
    \item \textbf{Two-Stage Architecture vượt trội}: Accuracy 97.19\% cao nhất.
    \item \textbf{Source Diversity Features hiệu quả}: 3 features mới cải thiện DDoS recall đáng kể.
    \item \textbf{SMOTE giải quyết imbalanced}: Normal recall đạt 99.95\%.
    \item \textbf{GPU Acceleration}: Tăng tốc 160x (12 phút vs 4 giờ).
\end{enumerate}

\textbf{DDoS Variants Classification}:
\begin{enumerate}
    \item \textbf{Accuracy cao}: 99.41\% trên test set với 2,863 mẫu, chỉ 13 misclassifications (HTTP $\leftrightarrow$ TCP)
    \item \textbf{Random Consecutive Sampling hiệu quả}: Preserve temporal patterns tốt hơn SMOTE, đạt phân phối hoàn hảo ~33.3\% cho mỗi biến thể DDoS
    \item \textbf{Feature selection tối ưu}: 16 best features đủ để phân biệt DDoS variants mà không cần full 35 features
    \item \textbf{Balanced performance}: Cả 3 biến thể DDoS đều đạt F1-score $>$ 99.2\%, trong đó UDP đạt 100\%
\end{enumerate}

\subsection{Hạn chế và thách thức}

\textbf{Two-Stage Model}:
\begin{enumerate}
    \item \textbf{DDoS Recall chưa hoàn hảo}: 93.22\%.
    \item \textbf{Theft category bị loại bỏ}: Do thiếu dữ liệu.
    \item \textbf{Memory Requirement cao}: 33GB RAM.
\end{enumerate}

\section{DEMO HỆ THỐNG PHÁT HIỆN DDOS}

Hệ thống demo phân tích theo lô (batch replay) các file flow xuất từ Argus.

\subsection{Kiến trúc hệ thống demo}

\subsubsection{Tổng quan kiến trúc}
Hệ thống hoạt động theo thời gian thực với luồng dữ liệu: Victim (packet capture) \textrightarrow{} Queue (inter-process communication) \textrightarrow{} Detector (3-stage ML pipeline) \textrightarrow{} Prometheus (time-series storage) \textrightarrow{} Grafana (dashboard). Các thành phần được triển khai trong cùng mạng LAN để mô phỏng kịch bản tấn công thực tế.

\subsubsection{Các thành phần hệ thống}
\begin{itemize}
    \item \textbf{Attacker}: Máy tấn công trong cùng LAN, sử dụng các công cụ như \texttt{hping3}, \texttt{slowhttptest}, \texttt{scapy} để thực hiện các cuộc tấn công DDoS (HTTP Flood, TCP SYN Flood, UDP Flood).
    \item \textbf{Victim Server}: Máy chủ đích trong cùng LAN, chạy service capture traffic bằng Scapy và đưa packets vào multiprocessing Queue. Có thể bật HTTP server để tiếp nhận HTTP Flood attacks.
    \item \textbf{Detector Server}: Nhận real-time packets từ Queue, xử lý qua 3-stage ML pipeline (Stage 1: Attack vs Normal, Stage 2: Attack Type, Stage 3: DDoS Variants), và export metrics sang Prometheus qua HTTP endpoint.
    \item \textbf{Prometheus (container)}: Thu thập và lưu trữ time-series metrics từ Detector tại \texttt{http://localhost:9090}.
    \item \textbf{Grafana (container)}: Truy vấn Prometheus và hiển thị real-time dashboard tại \texttt{http://localhost:3000}.
\end{itemize}

\subsubsection{Luồng dữ liệu thời gian thực}
\begin{itemize}
    \item \textbf{Capture}: Victim server sniff packets trên network interface (e.g., \texttt{eth0}, \texttt{wlan0}) bằng Scapy, thu thập theo batch (mặc định 1000 packets hoặc 30 giây timeout).
    \item \textbf{Inter-Process Communication}: Packets được gửi qua multiprocessing Queue từ Victim process sang Detector process (max queue size: 20 batches).
    \item \textbf{Feature Engineering}: Detector chuyển packets thành EventC objects, tính toán 22 features theo time windows (30 giây), bao gồm: packet rate, source entropy, bidirectional statistics, TCP flags, v.v.
    \item \textbf{Classification}: Mỗi time window qua 3-stage pipeline để phát hiện và phân loại tấn công.
    \item \textbf{Monitoring}: Metrics được cập nhật theo từng window và hiển thị trên Grafana dashboard (packet rate, source entropy, attack counts).
\end{itemize}

\subsection{Prometheus Metrics Specification}

\subsubsection{Giới thiệu Prometheus}
Prometheus làm kho time-series cho demo. Metrics được push qua HTTP; dữ liệu lưu kèm labels để truy vấn linh hoạt. Demo là offline replay; mỗi lần chạy lại sẽ cập nhật số liệu mới.

\subsubsection{Metric: ddos\_attack\_windows\_total (Counter)}
\begin{itemize}
    \item Labels: \texttt{attack\_type} (dos, ddos, reconnaissance), \texttt{ddos\_variant} (http, tcp, udp, normal), \texttt{dst\_ip}.
    \item Mục đích: Đếm số cửa sổ tấn công theo loại và biến thể.
\end{itemize}

\subsubsection{Metric: ddos\_normal\_windows\_total (Counter)}
\begin{itemize}
    \item Label: \texttt{dst\_ip}.
    \item Mục đích: Đếm số cửa sổ traffic bình thường (Stage 1 classify Normal).
\end{itemize}

\subsubsection{Metric: ddos\_packet\_rate (Gauge)}
\begin{itemize}
    \item Unit: packets per second (pps); Label: \texttt{dst\_ip}.
    \item Ý nghĩa: Cường độ traffic; ngưỡng tham chiếu: <0.5 (Normal), 0.5--2 (DoS), 2--10 (DDoS medium), >10 (DDoS high).
\end{itemize}

\subsubsection{Metric: ddos\_src\_entropy (Gauge)}
\begin{itemize}
    \item Unit: Shannon entropy (0--8); Label: \texttt{dst\_ip}.
    \item Ý nghĩa: Độ phân tán nguồn; entropy thấp (<1.5) gợi ý DoS/normal, entropy cao (>3) gợi ý DDoS phân tán.
\end{itemize}

\subsubsection{Ví dụ PromQL}
\begin{verbatim}
sum by (attack_type) (ddos_attack_windows_total)
sum by (ddos_variant) (ddos_attack_windows_total{attack_type="ddos"})
topk(15, max(ddos_src_entropy) by (dst_ip) > 1)
avg(ddos_packet_rate) by (dst_ip) > 0.5
sum(ddos_attack_windows_total)
\end{verbatim}

\subsection{Grafana Dashboard Configuration}

\subsubsection{Tổng quan Dashboard}
Dashboard "DDoS Detection Real-time Monitoring" hiển thị kết quả phát hiện theo thời gian thực với 5 panels chính. Dashboard tự động cập nhật để theo dõi attack patterns liên tục.

\begin{figure}[H]
    \centering
    \includegraphics[width=0.9\textwidth]{grafana_sc/Pannel_Overview.png}
    \caption{Tổng quan dashboard 5 panels}
    \label{fig:ddos_dashboard_overview}
\end{figure}

Nhận xét: Dashboard mới được tối ưu hóa với 5 panels:
\begin{itemize}
    \item \textbf{Panel 1 - Overview}: 5 stat cards hiển thị tổng quan tức thì (Total Attacks, DDoS, DoS, Normal, Reconnaissance)
    \item \textbf{Panel 2 - Attack Timeline}: Time series theo dõi attack rate theo thời gian (attacks/minute)
    \item \textbf{Panel 3 - DDoS Variant Breakdown}: Donut chart phân tích các biến thể DDoS (HTTP/TCP/UDP)
    \item \textbf{Panel 4 - Attack Type Distribution}: Bar chart phân bố loại tấn công (Stage 2 results)
    \item \textbf{Panel 5 - Attack vs Normal}: Pie chart so sánh tổng thể attack vs normal traffic
\end{itemize}
Bố cục này cho phép operator nắm bắt toàn diện từ tổng quan đến chi tiết trong real-time monitoring.

\subsubsection{Panel 1: Overview (Stat Cards)}
\begin{verbatim}
Total Attacks Detected:
  sum(ddos_attack_windows_total)
DDoS Attacks:
  sum(ddos_attack_windows_total{attack_type="ddos"})
DoS Attacks:
  sum(ddos_attack_windows_total{attack_type="dos"})
Normal Windows:
  sum(ddos_normal_windows_total)
Reconnaissance Attacks:
  sum(ddos_attack_windows_total{attack_type="reconnaissance"})
\end{verbatim}
\begin{figure}[H]
    \centering
    \includegraphics[width=0.9\textwidth]{grafana_sc/Overview.png}
    \caption{Overview - 5 Stat Cards}
    \label{fig:overview_stats}
\end{figure}
Nhận xét: Panel này cung cấp snapshot tức thì của tình hình hệ thống với 5 metrics chính:
\begin{itemize}
    \item \textbf{Total Attacks Detected}: Tổng số cửa sổ tấn công mà Stage 1 phát hiện được, KPI quan trọng nhất để đánh giá attack volume
    \item \textbf{DDoS Attacks}: Số cửa sổ tấn công phân tán từ nhiều nguồn (Stage 2 classify DDoS)
    \item \textbf{DoS Attacks}: Số cửa sổ tấn công từ ít nguồn (Stage 2 classify DoS)
    \item \textbf{Normal Windows}: Số cửa sổ traffic bình thường (Stage 1 classify Normal)
    \item \textbf{Reconnaissance Attacks}: Số cửa sổ quét thông tin (Stage 2 classify Recon)
\end{itemize}
Stat cards cho phép operator nhanh chóng đánh giá attack landscape trong vài giây. Màu sắc phân biệt rõ ràng giúp ưu tiên xử lý.

\subsubsection{Panel 2: Attack Timeline (Time Series)}
\begin{verbatim}
sum by (attack_type) (rate(ddos_attack_windows_total[1m]))
\end{verbatim}
\begin{figure}[H]
    \centering
    \includegraphics[width=0.9\textwidth]{grafana_sc/Attack Timeline.png}
    \caption{Attack Timeline - Real-time Attack Rate}
    \label{fig:attack_timeline}
\end{figure}
Nhận xét: Time series này hiển thị attack rate (attacks per second) theo thời gian với stacked area visualization. Mỗi attack type có màu riêng để dễ phân biệt:
\begin{itemize}
    \item \textbf{DoS}: Thường chiếm phần lớn diện tích khi có DoS attacks
    \item \textbf{DDoS}: Spike khi có DDoS campaigns từ nhiều nguồn
    \item \textbf{Reconnaissance}: Baseline thấp, spike ngắn khi có port scanning
\end{itemize}
Query sử dụng \texttt{rate()} với 1-minute window để smooth out noise và hiển thị trend rõ ràng. Panel này rất hữu ích để:
\begin{itemize}
    \item Phát hiện attack waves (các đợt tấn công lặp lại)
    \item So sánh cường độ tấn công giữa các thời điểm
    \item Xác định thời điểm peak attack để điều tra nguyên nhân
\end{itemize}
Operator có thể zoom in/out timeline để phân tích chi tiết hoặc xem tổng quan dài hạn.

\subsubsection{Panel 3: DDoS Variant Breakdown (Donut Chart)}
\begin{verbatim}
sum by (ddos_variant) (increase(ddos_attack_windows_total{attack_type="ddos"}[5m]))
\end{verbatim}
\begin{figure}[H]
    \centering
    \includegraphics[width=0.75\textwidth]{grafana_sc/DDoS Variant Breakdown.png}
    \caption{DDoS Variant Breakdown - Stage 3 Results}
    \label{fig:ddos_variant_breakdown}
\end{figure}
Nhận xét: Donut chart này thể hiện phân bố các biến thể DDoS trong 5 phút gần nhất, trực quan hóa kết quả từ Stage 3 classifier:
\begin{itemize}
    \item \textbf{HTTP}: DDoS-HTTP floods (slowloris, HTTP GET floods)
    \item \textbf{TCP}: DDoS-TCP attacks (SYN floods, ACK floods)
    \item \textbf{UDP}: DDoS-UDP floods (UDP amplification, random UDP)
\end{itemize}
Query sử dụng \texttt{increase()} với 5-minute window để tính số attack windows mới trong khoảng thời gian gần đây, filter chỉ lấy \texttt{attack\_type="ddos"}. Visualization này giúp:
\begin{itemize}
    \item Xác định attack vector chủ đạo để áp dụng mitigation strategy phù hợp
    \item So sánh tỷ lệ giữa các variants (ví dụ: nếu HTTP chiếm 80\% thì ưu tiên protect web server)
    \item Phát hiện multi-vector attacks (khi cả 3 variants đều có tỷ lệ đáng kể)
\end{itemize}
Màu sắc phân biệt rõ ràng giúp operator nhanh chóng nhận diện variant nguy hiểm nhất.

\subsubsection{Panel 4: Attack Type Distribution (Bar Chart)}
\begin{verbatim}
sum by (attack_type) (increase(ddos_attack_windows_total[5m]))
\end{verbatim}
\begin{figure}[H]
    \centering
    \includegraphics[width=0.8\textwidth]{grafana_sc/Attack Type Distribution.png}
    \caption{Attack Type Distribution - Stage 2 Results}
    \label{fig:attack_type_distribution_new}
\end{figure}
Nhận xét: Bar chart này hiển thị kết quả từ Stage 2 (phân loại loại tấn công) trong 5 phút gần nhất. Chiều cao các cột thể hiện số attack windows phát hiện được:
\begin{itemize}
    \item \textbf{DoS}: Single-source attacks, thường có bar cao nhất vì dễ thực hiện
    \item \textbf{DDoS}: Distributed attacks, bar cao cho thấy có botnet hoạt động
    \item \textbf{Reconnaissance}: Scanning/probing, bar thấp nhưng là early warning
\end{itemize}
Query sử dụng \texttt{increase()} với 5-minute window tương tự Panel 3, nhưng group by \texttt{attack\_type} thay vì \texttt{ddos\_variant}. Visualization này hữu ích để:
\begin{itemize}
    \item So sánh nhanh volume giữa DoS, DDoS, Reconnaissance
    \item Phát hiện shift trong attack strategy (ví dụ: từ DoS sang DDoS)
    \item Ưu tiên response team (DDoS có priority cao hơn DoS vì impact rộng hơn)
\end{itemize}
Bar chart dễ đọc hơn pie chart khi cần so sánh chính xác giữa các categories.

\subsubsection{Panel 5: Attack vs Normal (Pie Chart)}
\begin{verbatim}
Attack Windows:
  sum(increase(ddos_attack_windows_total[5m]))
Normal Windows:
  sum(increase(ddos_normal_windows_total[5m]))
\end{verbatim}
\begin{figure}[H]
    \centering
    \includegraphics[width=0.75\textwidth]{grafana_sc/Attack vs Normal.png}
    \caption{Attack vs Normal - Binary Classification Overview}
    \label{fig:attack_vs_normal}
\end{figure}
Nhận xét: Pie chart này cung cấp high-level view của Stage 1 binary classifier trong 5 phút gần nhất. Chỉ có 2 segments:
\begin{itemize}
    \item \textbf{Attack Windows}: Tổng tất cả attack windows (DDoS + DoS + Reconnaissance)
    \item \textbf{Normal Windows}: Traffic bình thường không có dấu hiệu tấn công
\end{itemize}
Tỷ lệ giữa hai segments cho thấy mức độ nghiêm trọng:
\begin{itemize}
    \item Normal > 90\%: Hệ thống an toàn, chỉ có noise traffic
    \item Attack 50--90\%: Đang bị tấn công, cần monitoring sát
    \item Attack > 90\%: Đang bị tấn công nặng, cần kích hoạt mitigation ngay
\end{itemize}
Query sử dụng \texttt{increase()} với 5-minute window để tính tổng số windows mới. Panel này là first checkpoint để đánh giá overall security posture. Nếu pie chart toàn màu đỏ, operator biết ngay hệ thống đang under attack và cần xem các panels chi tiết (2, 3, 4) để hiểu rõ attack type và variant.

\subsubsection{Tổng kết về Dashboard}
\begin{itemize}
    \item Dashboard tối ưu với 5 panels chính: Overview (5 stat cards), Attack Timeline, DDoS Variant Breakdown, Attack Type Distribution, Attack vs Normal.
    \item Real-time monitoring: tự động cập nhật với \texttt{rate()} và \texttt{increase()} queries.
    \item Màu sắc phân biệt rõ ràng giúp ưu tiên xử lý nhanh: đỏ (critical) > cam (warning) > vàng (attention) > xanh (normal).
    \item Workflow: Panel 5 (tổng quan) → Panel 1 (chi tiết metrics) → Panel 2 (timeline) → Panel 4 (attack type) → Panel 3 (DDoS variants).
\end{itemize}

\subsubsection{Tổng kết về Dashboard}
\begin{itemize}
    \item Ba lớp thông tin: (1) Overview Stat; (2) Distribution Charts; (3) Technical Gauges.
    \item Màu hoá: xanh/vàng/cam/đỏ giúp ưu tiên xử lý nhanh.
    \item Phù hợp quy trình batch: sau mỗi lượt replay, dashboard hiển thị ngay kết quả mới nhất.
\end{itemize}

\subsection{Phân tích và đánh giá}

\subsubsection{Hiệu quả giám sát (batch/near real-time khi chạy lặp)}
\begin{itemize}
    \item Quan sát kết quả ngay sau mỗi lượt replay batch; auto-refresh chỉ cập nhật số liệu batch mới.
    \item Lập lịch chạy định kỳ giúp tiệm cận near real-time ở mức vận hành thực tế.
    \item Entropy và packet rate hỗ trợ ưu tiên xử lý các mục tiêu bị tấn công nặng.
\end{itemize}

\subsubsection{Ưu điểm kiến trúc 3-stage + Prometheus/Grafana}
\begin{itemize}
    \item 3-stage: Stage 1 (Attack vs Normal), Stage 2 (DoS/DDoS/Recon), Stage 3 (HTTP/TCP/UDP cho DDoS).
    \item Prometheus: lưu trữ time-series hiệu quả, truy vấn linh hoạt bằng PromQL.
    \item Grafana: trực quan hoá đa dạng; dễ mở rộng thêm panels/metrics mới.
    \item Kiến trúc tách rời: Detector, Prometheus, Grafana độc lập, dễ thay thế/mở rộng.
\end{itemize}

\subsubsection{Nhận xét về kết quả demo}
\begin{itemize}
    \item Tổng \~19{,}174 windows: DoS 17{,}851 (93\%), DDoS 697 (3.6\%), Normal 627 (3.3\%), Recon 270 (1.4\%).
    \item Entropy: >3 thường gắn với DDoS; <1.5 gắn với DoS/normal.
    \item Packet rate: DoS 0.5--2 pps; DDoS >2 pps, nhiều trường hợp >10 pps.
    \item Stage 1 phát hiện gần 98\% attack windows trong kịch bản replay.
\end{itemize}

\section{Kết luận chương}
Chương này trình bày kết quả thực nghiệm của hai hướng nghiên cứu chính:

\textbf{Two-Stage Hierarchical Model} đạt accuracy 97.19\% với kiến trúc hai giai đoạn, chứng minh hiệu quả của Binary Classification + Multi-class Classification kết hợp Source Diversity Features và SMOTE. Hệ thống phát hiện botnet đa dạng (DDoS, DoS, Reconnaissance) với Normal recall 99.95\% và training time chỉ 12 phút nhờ GPU acceleration.

\textbf{DDoS Variants Classification} đạt accuracy 99.41\% với XGBoost Multi-class, phân loại chi tiết 3 biến thể DDoS (HTTP, TCP, UDP) với balanced performance trên cả 3 classes (F1-score từ 99.25\% đến 100\%). Random Consecutive Sampling preserve temporal patterns hiệu quả, 16 best features tối ưu tài nguyên, và training chỉ $<$2 phút trên CPU thường (không cần GPU).

Cả hai mô hình đều chứng minh tính khả thi triển khai thực tế với performance cao và resource requirements hợp lý.


\chapter{KẾT LUẬN VÀ HƯỚNG PHÁT TRIỂN}

\section{Tóm tắt kết quả}

Nghiên cứu đã xây dựng thành công hệ thống phát hiện và phân loại botnet IoT với hai mô hình chính:

\textbf{Two-Stage Hierarchical Model} đạt Overall Accuracy \textbf{97.19\%} trên tập dữ liệu Bot-IoT với 100,000 samples cân bằng:
\begin{itemize}
    \item Stage 1 (Binary): Accuracy 99.26\%, ROC-AUC 99.99\%
    \item Stage 2 (Multi-class): Accuracy 97.58\%, F1-score weighted 97.64\%
    \item Training time: 12 phút (GPU T4)
\end{itemize}

\textbf{DDoS Variants Classification} đạt Accuracy \textbf{99.41\%} trên bài toán phân loại chi tiết các biến thể DDoS:
\begin{itemize}
    \item Overall Metrics: Precision 99.48\%, Recall 99.36\%, F1-Score 99.42\%
    \item Per-class F1-score: HTTP 99.35\%, TCP 99.25\%, UDP 100\%
    \item Training time: $<$2 phút trên CPU (không cần GPU)
\end{itemize}


\section{Đóng góp chính}

\textbf{1. Kiến trúc Three-Stage Hierarchical Model}

Đề xuất kiến trúc phân cấp ba giai đoạn: Binary Classification → Multi-class Attack Type → DDoS Variant Classification. Kiến trúc này cho phép tối ưu hóa riêng biệt cho từng bài toán con.

\textbf{2. Source Diversity Features}

Thiết kế 3 đặc trưng mới (\texttt{unique\_src\_count}, \texttt{src\_entropy}, \texttt{top\_src\_ratio}) giúp phân biệt hiệu quả DDoS (nhiều nguồn) và DoS (ít nguồn).

\textbf{3. Random Consecutive Sampling}

Phương pháp cân bằng dữ liệu mới thay thế SMOTE cho DDoS variants, bảo toàn temporal patterns và tránh synthetic artifacts. Đạt perfect class balance 25\% mỗi class.

\textbf{4. Balanced Test Set và Evaluation Framework}

Xây dựng balanced test set 100K samples với distribution thực tế (DoS 50\%, DDoS 35\%, Recon 13\%, Normal 2\%) để đánh giá công bằng hiệu năng model.

\section{Hạn chế}

\textbf{Về hiệu năng:}
\begin{itemize}
    \item DDoS recall (93.22\%) thấp hơn so với DoS và Reconnaissance (99\%+), do sự tương đồng giữa low-rate DDoS và Normal traffic
    \item Theft category bị loại bỏ do chỉ có 1,587 samples (0.003\%)
\end{itemize}

\textbf{Về tài nguyên:}
\begin{itemize}
    \item Training cần 33GB RAM, yêu cầu cloud instances với High-RAM
    \item Chưa tối ưu cho edge devices với RAM hạn chế
\end{itemize}

\textbf{Về validation:}
\begin{itemize}
    \item Chỉ validate trên Bot-IoT dataset (lab environment)
    \item Chưa test trên real-world traffic và các datasets khác (CICIDS, NSL-KDD)
\end{itemize}

\section{Hướng phát triển}

\textbf{Cải thiện model:}
\begin{itemize}
    \item Temporal Feature Engineering: Thêm time-series features (flow rate, inter-arrival time)
    \item Ensemble với Deep Learning: Kết hợp XGBoost với CNN-LSTM
    \item Adversarial robustness và Explainable AI (SHAP, LIME)
\end{itemize}

\textbf{Triển khai thực tế:}
\begin{itemize}
    \item REST API với Flask/FastAPI cho real-time detection
    \item Edge deployment trên IoT gateway (Raspberry Pi, NVIDIA Jetson)
    \item Integration với SIEM systems và SOC workflows
\end{itemize}

\textbf{Mở rộng nghiên cứu:}
\begin{itemize}
    \item Cross-dataset validation (CICIDS, NSL-KDD, UNSW-NB15)
    \item Online learning để adapt với new attack patterns
    \item Model compression (quantization, pruning) cho deployment nhẹ hơn
\end{itemize}

\section{Kết luận}

Nghiên cứu đã đạt được các mục tiêu đề ra: xây dựng hệ thống IDS hiệu quả cho IoT với accuracy >97\%, training time <15 phút, và khả năng phân loại chi tiết các biến thể DDoS đạt 99.92\%. Các đóng góp chính bao gồm kiến trúc Three-Stage, Source Diversity Features, và Random Consecutive Sampling. Kết quả nghiên cứu đặt nền móng vững chắc cho phát triển giải pháp bảo mật IoT toàn diện với khả năng triển khai thực tế cao.

\addcontentsline{toc}{chapter}{TÀI LIỆU THAM KHẢO}
\begin{thebibliography}{99}

\bibitem{ref1} N. Koroniotis, N. Moustafa, E. Sitnikova, and B. Turnbull, "Towards the development of realistic botnet dataset in the Internet of Things for network forensic analytics: Bot-IoT dataset," \textit{Future Generation Computer Systems}, vol. 100, pp. 779–796, 2019.

\bibitem{ref2} M. Sarhan, S. Layeghy, N. Moustafa, and M. Portmann, "NetFlow datasets for machine learning-based network intrusion detection systems," \textit{arXiv preprint arXiv:2011.09144}, 2020.

\bibitem{ref3} S. Alosaimi and S. M. Almutairi, "An intrusion detection system using BoT-IoT," \textit{Applied Sciences}, vol. 13, no. 9, art. 5427, 2023, doi: 10.3390/app13095427.

\bibitem{ref4} F. H. Zawaideh, G. Al-Asad, G. Swaneh, S. Batainah, and H. Bakkar, "Intrusion detection system for IoT networks using convolutional neural network (CNN) and XGBoost algorithm," \textit{Journal of Theoretical and Applied Information Technology}, vol. 102, no. 4, 2024.

\bibitem{ref5} H. Zhang, B. Zhang, L. Huang, Z. Zhang, and H. Huang, "An efficient two-stage network intrusion detection system in the Internet of Things," \textit{Information}, vol. 14, no. 2, art. 77, 2023, doi: 10.3390/info14020077.

\bibitem{ref6} E. Özdoğan, O. Ceran, M. Uysal, M. T. Üstündağ, and M. R. Habib, "IoT intrusion detection: Implementing a dual-layered security approach," \textit{International Journal of Intelligent Systems}, vol. 2025, art. 8884584, pp. 1–23, 2025, doi: 10.1155/int.8884584.

\bibitem{ref7} T.-T.-H. Le, Y. E. Oktian, and H. Kim, "XGBoost for imbalanced multiclass classification-based industrial Internet of Things intrusion detection systems," \textit{Sustainability}, vol. 14, no. 14, art. 8707, 2022, doi: 10.3390/su14148707.

\bibitem{ref8} D. F. Doghramachi and S. Y. Ameen, "Internet of Things (IoT) security enhancement using XGBoost machine learning techniques," \textit{Computers, Materials \& Continua}, vol. 77, no. 1, pp. 717–732, 2023, doi: 10.32604/cmc.2023.041186.

\bibitem{ref9} S. Chalichalamala, N. Govindan, and R. Kasarapu, "An extreme gradient boost based classification and regression tree for network intrusion detection in IoT," \textit{Bulletin of Electrical Engineering and Informatics}, vol. 13, no. 3, 2024, doi: 10.11591/eei.v13i3.6843.

\bibitem{ref10} M. R. A. Khan, A. Y. Barnawi, A. Munir, Z. Alsalman, and D. M. S. Sanunga, "Lightweight quantized XGBoost for botnet detection in resource-constrained IoT networks," \textit{IoT}, vol. 6, no. 4, art. 70, 2025, doi: 10.3390/iot6040070.

\bibitem{ref11} T. Chen and C. Guestrin, "XGBoost: A scalable tree boosting system," in \textit{Proceedings of the 22nd ACM SIGKDD International Conference on Knowledge Discovery and Data Mining}, San Francisco, CA, USA, 2016, pp. 785-794.

\bibitem{ref12} J. H. Friedman, "Greedy function approximation: A gradient boosting machine," \textit{The Annals of Statistics}, vol. 29, no. 5, pp. 1189-1232, 2001.

\bibitem{ref13} N. V. Chawla, K. W. Bowyer, L. O. Hall, and W. P. Kegelmeyer, "SMOTE: Synthetic minority over-sampling technique," \textit{Journal of Artificial Intelligence Research}, vol. 16, pp. 321-357, 2002.

\bibitem{ref14} H. He and E. A. Garcia, "Learning from imbalanced data," \textit{IEEE Transactions on Knowledge and Data Engineering}, vol. 21, no. 9, pp. 1263-1284, 2009.

\bibitem{ref15} A. Fernández, S. García, M. Galar, R. C. Prati, B. Krawczyk, and F. Herrera, \textit{Learning from Imbalanced Data Sets}, Springer International Publishing, 2018.

\bibitem{ref16} C. N. Silla Jr. and A. A. Freitas, "A survey of hierarchical classification across different application domains," \textit{Data Mining and Knowledge Discovery}, vol. 22, no. 1-2, pp. 31-72, 2011.

\bibitem{ref17} C. Kolias, G. Kambourakis, A. Stavrou, and J. Voas, "DDoS in the IoT: Mirai and other botnets," \textit{Computer}, vol. 50, no. 7, pp. 80-84, 2017.

\bibitem{ref18} M. Antonakakis et al., "Understanding the Mirai botnet," in \textit{Proceedings of the 26th USENIX Security Symposium}, Vancouver, BC, Canada, 2017, pp. 1093-1110.

\bibitem{ref19} A. L. Buczak and E. Guven, "A survey of data mining and machine learning methods for cyber security intrusion detection," \textit{IEEE Communications Surveys \& Tutorials}, vol. 18, no. 2, pp. 1153-1176, 2016.

\bibitem{ref20} A. Khraisat, I. Gondal, P. Vamplew, and J. Kamruzzaman, "Survey of intrusion detection systems: techniques, datasets and challenges," \textit{Cybersecurity}, vol. 2, no. 1, pp. 1-22, 2019.

\bibitem{ref21} R. Vinayakumar, M. Alazab, K. P. Soman, P. Poornachandran, A. Al-Nemrat, and S. Venkatraman, "Deep learning approach for intelligent intrusion detection system," \textit{IEEE Access}, vol. 7, pp. 41525-41550, 2019.

\bibitem{ref22} T. A. Tang, L. Mhamdi, D. McLernon, S. A. R. Zaidi, and M. Ghogho, "Deep learning approach for network intrusion detection in software defined networking," in \textit{Proceedings of the International Conference on Wireless Networks and Mobile Communications (WINCOM)}, Fez, Morocco, 2016, pp. 258-263.

\bibitem{ref23} Google Colaboratory, "Google Colab Pro+," Google Research, 2023. [Online]. Available: https://colab.research.google.com/

\bibitem{ref24} NVIDIA Corporation, "CUDA Toolkit Documentation," NVIDIA Developer, 2023. [Online]. Available: https://docs.nvidia.com/cuda/

\bibitem{ref25} Statista Research Department, "IoT: Number of connected devices worldwide 2015-2025," \textit{Statista}, 2023. [Online]. Available: https://www.statista.com/statistics/471264/iot-number-of-connected-devices-worldwide/

\bibitem{ref26} A. Gutnikov, O. Kupreev, and Y. Sintsiak, "DDoS attacks in Q3 2021," \textit{Securelist}, Kaspersky, 2021. [Online]. Available: https://securelist.com/ddos-attacks-in-q3-2021/104796/

\bibitem{ref27} "GoldenEye Layer 7 (KeepAlive+NoCache) DoS Test Tool," GitHub Repository. [Online]. Available: https://github.com/jseidl/GoldenEye

\bibitem{ref28} "Hping -- Active Network Security Tool," 2021. [Online]. Available: http://www.hping.org

\end{thebibliography}

\end{document}





